%% Generated by Sphinx.
\def\sphinxdocclass{report}
\documentclass[letterpaper,10pt,english]{sphinxmanual}
\ifdefined\pdfpxdimen
   \let\sphinxpxdimen\pdfpxdimen\else\newdimen\sphinxpxdimen
\fi \sphinxpxdimen=.75bp\relax

\PassOptionsToPackage{warn}{textcomp}
\usepackage[utf8]{inputenc}
\ifdefined\DeclareUnicodeCharacter
% support both utf8 and utf8x syntaxes
  \ifdefined\DeclareUnicodeCharacterAsOptional
    \def\sphinxDUC#1{\DeclareUnicodeCharacter{"#1}}
  \else
    \let\sphinxDUC\DeclareUnicodeCharacter
  \fi
  \sphinxDUC{00A0}{\nobreakspace}
  \sphinxDUC{2500}{\sphinxunichar{2500}}
  \sphinxDUC{2502}{\sphinxunichar{2502}}
  \sphinxDUC{2514}{\sphinxunichar{2514}}
  \sphinxDUC{251C}{\sphinxunichar{251C}}
  \sphinxDUC{2572}{\textbackslash}
\fi
\usepackage{cmap}
\usepackage[T1]{fontenc}
\usepackage{amsmath,amssymb,amstext}
\usepackage{babel}



\usepackage{times}
\expandafter\ifx\csname T@LGR\endcsname\relax
\else
% LGR was declared as font encoding
  \substitutefont{LGR}{\rmdefault}{cmr}
  \substitutefont{LGR}{\sfdefault}{cmss}
  \substitutefont{LGR}{\ttdefault}{cmtt}
\fi
\expandafter\ifx\csname T@X2\endcsname\relax
  \expandafter\ifx\csname T@T2A\endcsname\relax
  \else
  % T2A was declared as font encoding
    \substitutefont{T2A}{\rmdefault}{cmr}
    \substitutefont{T2A}{\sfdefault}{cmss}
    \substitutefont{T2A}{\ttdefault}{cmtt}
  \fi
\else
% X2 was declared as font encoding
  \substitutefont{X2}{\rmdefault}{cmr}
  \substitutefont{X2}{\sfdefault}{cmss}
  \substitutefont{X2}{\ttdefault}{cmtt}
\fi


\usepackage[Bjarne]{fncychap}
\usepackage{sphinx}

\fvset{fontsize=\small}
\usepackage{geometry}


% Include hyperref last.
\usepackage{hyperref}
% Fix anchor placement for figures with captions.
\usepackage{hypcap}% it must be loaded after hyperref.
% Set up styles of URL: it should be placed after hyperref.
\urlstyle{same}

\addto\captionsenglish{\renewcommand{\contentsname}{Contents:}}

\usepackage{sphinxmessages}
\setcounter{tocdepth}{1}



\title{3K04\sphinxhyphen{}PacemakerProject DCM (Group 10)}
\date{Oct 29, 2020}
\release{}
\author{Graham Power, Hamza Ashraf}
\newcommand{\sphinxlogo}{\vbox{}}
\renewcommand{\releasename}{}
\makeindex
\begin{document}

\pagestyle{empty}
\sphinxmaketitle
\pagestyle{plain}
\sphinxtableofcontents
\pagestyle{normal}
\phantomsection\label{\detokenize{index::doc}}



\chapter{flaskapp}
\label{\detokenize{modules:flaskapp}}\label{\detokenize{modules::doc}}

\section{flaskapp package}
\label{\detokenize{flaskapp:flaskapp-package}}\label{\detokenize{flaskapp::doc}}

\subsection{Subpackages}
\label{\detokenize{flaskapp:subpackages}}

\subsubsection{flaskapp.config package}
\label{\detokenize{flaskapp.config:flaskapp-config-package}}\label{\detokenize{flaskapp.config::doc}}

\paragraph{Submodules}
\label{\detokenize{flaskapp.config:submodules}}

\paragraph{flaskapp.config.config\_manager module}
\label{\detokenize{flaskapp.config:module-flaskapp.config.config_manager}}\label{\detokenize{flaskapp.config:flaskapp-config-config-manager-module}}\index{module@\spxentry{module}!flaskapp.config.config\_manager@\spxentry{flaskapp.config.config\_manager}}\index{flaskapp.config.config\_manager@\spxentry{flaskapp.config.config\_manager}!module@\spxentry{module}}

\subparagraph{Config Manager}
\label{\detokenize{flaskapp.config:config-manager}}
A collection of configuration setup
functions, to initialize the config
variables and setup application
logging.
\index{init\_config() (in module flaskapp.config.config\_manager)@\spxentry{init\_config()}\spxextra{in module flaskapp.config.config\_manager}}

\begin{fulllineitems}
\phantomsection\label{\detokenize{flaskapp.config:flaskapp.config.config_manager.init_config}}\pysiglinewithargsret{\sphinxcode{\sphinxupquote{flaskapp.config.config\_manager.}}\sphinxbfcode{\sphinxupquote{init\_config}}}{\emph{\DUrole{n}{cfg\_files}\DUrole{o}{=}\DUrole{default_value}{{[}{]}}}}{}
init\_config Initializes the application configuration given a list of configuration files

Will read all configuration files, creating a application varaible matching the value in the
files. File locations in the list must be specified relative to the \textasciitilde{}/3K04\sphinxhyphen{}Pacemaker/DCM/flaskapp/config
directory. The order of the files in the list matters, as any repeat variables will have their previous
values overriden. This allows for a default application configuration to be specified and later overriden
by an end user.
\begin{quote}\begin{description}
\item[{Parameters}] \leavevmode
\sphinxstyleliteralstrong{\sphinxupquote{cfg\_files}} (\sphinxstyleliteralemphasis{\sphinxupquote{list}}\sphinxstyleliteralemphasis{\sphinxupquote{, }}\sphinxstyleliteralemphasis{\sphinxupquote{optional}}) \textendash{} A list of config file locations, defaults to {[}{]}

\item[{Returns}] \leavevmode
A handler for the applications configuration manager

\item[{Return type}] \leavevmode
\sphinxcode{\sphinxupquote{configparser.ConfigParser}}

\end{description}\end{quote}

\end{fulllineitems}

\index{init\_logging() (in module flaskapp.config.config\_manager)@\spxentry{init\_logging()}\spxextra{in module flaskapp.config.config\_manager}}

\begin{fulllineitems}
\phantomsection\label{\detokenize{flaskapp.config:flaskapp.config.config_manager.init_logging}}\pysiglinewithargsret{\sphinxcode{\sphinxupquote{flaskapp.config.config\_manager.}}\sphinxbfcode{\sphinxupquote{init\_logging}}}{\emph{\DUrole{n}{config}}}{}
init\_logging Initializes the application logging

Will create two logging handlers, a file and stream hendler. This ensures all
log output is sent to both the console at runtime as well as a dated log file
located in the /logs directory. Both handlers are created as defined in the
Logging section of the application configuration files.
\begin{quote}\begin{description}
\item[{Parameters}] \leavevmode
\sphinxstyleliteralstrong{\sphinxupquote{config}} (\sphinxcode{\sphinxupquote{configparser.ConfigParser}}) \textendash{} The handler for the application configuration

\item[{Returns}] \leavevmode
A handler for the applications logging manager

\item[{Return type}] \leavevmode
\sphinxcode{\sphinxupquote{logging.Logger}}

\end{description}\end{quote}

\end{fulllineitems}



\paragraph{flaskapp.config.decorators module}
\label{\detokenize{flaskapp.config:module-flaskapp.config.decorators}}\label{\detokenize{flaskapp.config:flaskapp-config-decorators-module}}\index{module@\spxentry{module}!flaskapp.config.decorators@\spxentry{flaskapp.config.decorators}}\index{flaskapp.config.decorators@\spxentry{flaskapp.config.decorators}!module@\spxentry{module}}

\subparagraph{Decorators Library}
\label{\detokenize{flaskapp.config:decorators-library}}
A collection of decorators used by the main
app, created to avoid the unnecessary
reproduction of code.
\index{login\_required() (in module flaskapp.config.decorators)@\spxentry{login\_required()}\spxextra{in module flaskapp.config.decorators}}

\begin{fulllineitems}
\phantomsection\label{\detokenize{flaskapp.config:flaskapp.config.decorators.login_required}}\pysiglinewithargsret{\sphinxcode{\sphinxupquote{flaskapp.config.decorators.}}\sphinxbfcode{\sphinxupquote{login\_required}}}{\emph{\DUrole{n}{f}}}{}
login\_required Ensures that a user is logged in before granting access to user restricted pages

Will check if a user is logged in and only allow a redirect to the user restricted page if there is
a user logged in. If no user is logged in it will chenge the redirect to the home page and flash a
login required messege, letting the user know they must login before accessing that endpoint.
NOTE: This function should never be called as a function, only used as a decorator above functions
who require its functionality to be implemented on entry. Always use a decorator (i.e. @login\_required)
to invoke this function.
\begin{quote}\begin{description}
\item[{Parameters}] \leavevmode
\sphinxstyleliteralstrong{\sphinxupquote{f}} (\sphinxstyleliteralemphasis{\sphinxupquote{function}}) \textendash{} The function being decorated. This will be automatically filled when using the @ tag

\item[{Returns}] \leavevmode
The wrap function, whose contents are the input function, modified to add the functionality of this decorator

\item[{Return type}] \leavevmode
function

\end{description}\end{quote}

\end{fulllineitems}

\index{logout\_required() (in module flaskapp.config.decorators)@\spxentry{logout\_required()}\spxextra{in module flaskapp.config.decorators}}

\begin{fulllineitems}
\phantomsection\label{\detokenize{flaskapp.config:flaskapp.config.decorators.logout_required}}\pysiglinewithargsret{\sphinxcode{\sphinxupquote{flaskapp.config.decorators.}}\sphinxbfcode{\sphinxupquote{logout\_required}}}{\emph{\DUrole{n}{f}}}{}
logout\_required Ensures that a user is logged out before granting access to non user restricted pages

Will check if a user is logged out and only allow a redirect to the non user restricted page if there is
no user logged in. If a user is logged in it will simply log them out automatically before redirecting to
the requested endpoint. No logout action on the part of the user is necessary.
NOTE: This function should never be called as a function, only used as a decorator above functions
who require its functionality to be implemented on entry. Always use a decorator (i.e. @logout\_required)
to invoke this function.
\begin{quote}\begin{description}
\item[{Parameters}] \leavevmode
\sphinxstyleliteralstrong{\sphinxupquote{f}} (\sphinxstyleliteralemphasis{\sphinxupquote{function}}) \textendash{} The function being decorated. This will be automatically filled when using the @ tag

\item[{Returns}] \leavevmode
The wrap function, whose contents are the input function, modified to add the functionality of this decorator

\item[{Return type}] \leavevmode
function

\end{description}\end{quote}

\end{fulllineitems}



\paragraph{Module contents}
\label{\detokenize{flaskapp.config:module-flaskapp.config}}\label{\detokenize{flaskapp.config:module-contents}}\index{module@\spxentry{module}!flaskapp.config@\spxentry{flaskapp.config}}\index{flaskapp.config@\spxentry{flaskapp.config}!module@\spxentry{module}}

\subsubsection{flaskapp.data package}
\label{\detokenize{flaskapp.data:flaskapp-data-package}}\label{\detokenize{flaskapp.data::doc}}

\paragraph{Submodules}
\label{\detokenize{flaskapp.data:submodules}}

\paragraph{flaskapp.data.database module}
\label{\detokenize{flaskapp.data:module-flaskapp.data.database}}\label{\detokenize{flaskapp.data:flaskapp-data-database-module}}\index{module@\spxentry{module}!flaskapp.data.database@\spxentry{flaskapp.data.database}}\index{flaskapp.data.database@\spxentry{flaskapp.data.database}!module@\spxentry{module}}

\subparagraph{Database Library}
\label{\detokenize{flaskapp.data:database-library}}
A collection of functions capable of interacting with
a sqlite3 single file databse.
NOTE: Users are returned as lists, whose entries are
in the following order
\begin{quote}

{[} \_userid,
username,
password,
LowerRateLimit,
UpperRateLimit,
AtrialAmplitude,
AtrialPulseWidth,
AtrialRefractoryPeriod,
VentricularAmplitude,
VentricularPulseWidth,
VentricularRefractoryPeriod {]}
\end{quote}
\index{find\_user() (in module flaskapp.data.database)@\spxentry{find\_user()}\spxextra{in module flaskapp.data.database}}

\begin{fulllineitems}
\phantomsection\label{\detokenize{flaskapp.data:flaskapp.data.database.find_user}}\pysiglinewithargsret{\sphinxcode{\sphinxupquote{flaskapp.data.database.}}\sphinxbfcode{\sphinxupquote{find\_user}}}{\emph{\DUrole{n}{cursor}}, \emph{\DUrole{n}{username}\DUrole{o}{=}\DUrole{default_value}{None}}, \emph{\DUrole{n}{password}\DUrole{o}{=}\DUrole{default_value}{None}}}{}
find\_user Given search parameters will find all matching users in the database

Given one or more of the optional search parameters will return a list of all users
matching that search criteria. Accepted search parameters are username and password.
If neither optional parameters are given, the function will return None
\begin{quote}\begin{description}
\item[{Parameters}] \leavevmode\begin{itemize}
\item {} 
\sphinxstyleliteralstrong{\sphinxupquote{cursor}} (\sphinxcode{\sphinxupquote{sqlite3.Cursor}}) \textendash{} The cursor handler for the database to search for the user in

\item {} 
\sphinxstyleliteralstrong{\sphinxupquote{username}} (\sphinxstyleliteralemphasis{\sphinxupquote{str}}\sphinxstyleliteralemphasis{\sphinxupquote{, }}\sphinxstyleliteralemphasis{\sphinxupquote{optional}}) \textendash{} The username of the user to search for, defaults to None

\item {} 
\sphinxstyleliteralstrong{\sphinxupquote{password}} (\sphinxstyleliteralemphasis{\sphinxupquote{str}}\sphinxstyleliteralemphasis{\sphinxupquote{, }}\sphinxstyleliteralemphasis{\sphinxupquote{optional}}) \textendash{} The password of the user to search for, defaults to None

\end{itemize}

\item[{Returns}] \leavevmode
A list of tuples containing all users matching the search query

\item[{Return type}] \leavevmode
list

\end{description}\end{quote}

\end{fulllineitems}

\index{get\_rows() (in module flaskapp.data.database)@\spxentry{get\_rows()}\spxextra{in module flaskapp.data.database}}

\begin{fulllineitems}
\phantomsection\label{\detokenize{flaskapp.data:flaskapp.data.database.get_rows}}\pysiglinewithargsret{\sphinxcode{\sphinxupquote{flaskapp.data.database.}}\sphinxbfcode{\sphinxupquote{get\_rows}}}{\emph{\DUrole{n}{cursor}}}{}
get\_rows Returns the number of rows (.i.e users) in the database
\begin{quote}\begin{description}
\item[{Parameters}] \leavevmode
\sphinxstyleliteralstrong{\sphinxupquote{cursor}} (\sphinxcode{\sphinxupquote{sqlite3.Cursor}}) \textendash{} The cursor handler for the database

\item[{Returns}] \leavevmode
A list of tuples containing all items matching the search query

\item[{Return type}] \leavevmode
list

\end{description}\end{quote}

\end{fulllineitems}

\index{get\_user() (in module flaskapp.data.database)@\spxentry{get\_user()}\spxextra{in module flaskapp.data.database}}

\begin{fulllineitems}
\phantomsection\label{\detokenize{flaskapp.data:flaskapp.data.database.get_user}}\pysiglinewithargsret{\sphinxcode{\sphinxupquote{flaskapp.data.database.}}\sphinxbfcode{\sphinxupquote{get\_user}}}{\emph{\DUrole{n}{cursor}}, \emph{\DUrole{n}{id}}}{}
get\_user Reuturns a complete users information given their unique ID

Return type is a list of users. Since the search is done by unique ID,
this list is garunteed to be either of length one, if a user with matching
unique ID is found, or zero, if no user with matching unique ID is found.
\begin{quote}\begin{description}
\item[{Parameters}] \leavevmode\begin{itemize}
\item {} 
\sphinxstyleliteralstrong{\sphinxupquote{cursor}} (\sphinxcode{\sphinxupquote{sqlite3.Cursor}}) \textendash{} The cursor handler for the database the user can be found in

\item {} 
\sphinxstyleliteralstrong{\sphinxupquote{id}} (\sphinxstyleliteralemphasis{\sphinxupquote{int}}) \textendash{} The unique ID of the user to search for

\end{itemize}

\item[{Returns}] \leavevmode
A list of tuples containing all items matching the search query

\item[{Return type}] \leavevmode
list

\end{description}\end{quote}

\end{fulllineitems}

\index{init\_db() (in module flaskapp.data.database)@\spxentry{init\_db()}\spxextra{in module flaskapp.data.database}}

\begin{fulllineitems}
\phantomsection\label{\detokenize{flaskapp.data:flaskapp.data.database.init_db}}\pysiglinewithargsret{\sphinxcode{\sphinxupquote{flaskapp.data.database.}}\sphinxbfcode{\sphinxupquote{init\_db}}}{\emph{\DUrole{n}{file}}}{}
init\_db Initializes a database located at a given file location

The file location should be specified relative to the \textasciitilde{}/3K04\sphinxhyphen{}Pacemaker/DCM/flaskapp/data
directory. The file should also have a supported sqlite3 extension (.db .db3 .sdb .s3db
.sqlite .sqlite3) and if the file does not already exist it will be created and
populated with a new databases.
\begin{quote}\begin{description}
\item[{Parameters}] \leavevmode
\sphinxstyleliteralstrong{\sphinxupquote{file}} (\sphinxstyleliteralemphasis{\sphinxupquote{str}}) \textendash{} The relative file location of the single file sqlite3 database

\item[{Returns}] \leavevmode
A tuple containing the databases connection handler and cursor (sqlite3.Connection, sqlite3.Cursor)

\item[{Return type}] \leavevmode
tuple

\end{description}\end{quote}

\end{fulllineitems}

\index{insert\_user() (in module flaskapp.data.database)@\spxentry{insert\_user()}\spxextra{in module flaskapp.data.database}}

\begin{fulllineitems}
\phantomsection\label{\detokenize{flaskapp.data:flaskapp.data.database.insert_user}}\pysiglinewithargsret{\sphinxcode{\sphinxupquote{flaskapp.data.database.}}\sphinxbfcode{\sphinxupquote{insert\_user}}}{\emph{\DUrole{n}{conn}}, \emph{\DUrole{n}{cursor}}, \emph{\DUrole{n}{username}}, \emph{\DUrole{n}{password}}}{}
insert\_user Given a username and password of a new user, will insert the user into the database

This function will create a new entry in the database of a user with the given username and password.
Only the users username and password are initialized upon user creation, the pacemaker parameters will
default to None, forcing the user to manually enter their parameters.
NOTE: This function does no check for conflicting users in the database before inserting a new user.
It is up to the user of this function to check for conflicts (if they wish to do so) before calling
this function.
\begin{quote}\begin{description}
\item[{Parameters}] \leavevmode\begin{itemize}
\item {} 
\sphinxstyleliteralstrong{\sphinxupquote{conn}} (\sphinxcode{\sphinxupquote{sqlite3.Connection}}) \textendash{} The connection handler for the database to insert a new user into

\item {} 
\sphinxstyleliteralstrong{\sphinxupquote{cursor}} (\sphinxcode{\sphinxupquote{sqlite3.Cursor}}) \textendash{} The cursor handler for the database to insert a new user into

\item {} 
\sphinxstyleliteralstrong{\sphinxupquote{username}} (\sphinxstyleliteralemphasis{\sphinxupquote{str}}) \textendash{} The username for the new user

\item {} 
\sphinxstyleliteralstrong{\sphinxupquote{password}} (\sphinxstyleliteralemphasis{\sphinxupquote{str}}) \textendash{} The password for the new user

\end{itemize}

\end{description}\end{quote}

\end{fulllineitems}

\index{update\_pacemaker\_parameters() (in module flaskapp.data.database)@\spxentry{update\_pacemaker\_parameters()}\spxextra{in module flaskapp.data.database}}

\begin{fulllineitems}
\phantomsection\label{\detokenize{flaskapp.data:flaskapp.data.database.update_pacemaker_parameters}}\pysiglinewithargsret{\sphinxcode{\sphinxupquote{flaskapp.data.database.}}\sphinxbfcode{\sphinxupquote{update\_pacemaker\_parameters}}}{\emph{\DUrole{n}{conn}}, \emph{\DUrole{n}{cursor}}, \emph{\DUrole{n}{id}}, \emph{\DUrole{n}{values}}}{}
update\_pacemaker\_parameters Given a list of pacemaker parameters, updates the database values

When given handler to the database and the unique ID of the user being affected, will update the
users pacemaker parameters to match the input list.
NOTE: No complete check is done to ensure the validity of the input, it is up
to the method user to ensure the lists correctness.
\begin{quote}\begin{description}
\item[{Parameters}] \leavevmode\begin{itemize}
\item {} 
\sphinxstyleliteralstrong{\sphinxupquote{conn}} (\sphinxcode{\sphinxupquote{sqlite3.Connection}}) \textendash{} The connection handler for the database whos contents to change

\item {} 
\sphinxstyleliteralstrong{\sphinxupquote{cursor}} (\sphinxcode{\sphinxupquote{sqlite3.Cursor}}) \textendash{} The cursor handler for the database whos contents to change

\item {} 
\sphinxstyleliteralstrong{\sphinxupquote{id}} (\sphinxstyleliteralemphasis{\sphinxupquote{int}}) \textendash{} The unique ID of the user whos parameters should be changed

\item {} 
\sphinxstyleliteralstrong{\sphinxupquote{values}} (\sphinxstyleliteralemphasis{\sphinxupquote{list}}) \textendash{} A list of pacemaker parameters, whos order matches the databases contents

\end{itemize}

\end{description}\end{quote}

\end{fulllineitems}



\paragraph{flaskapp.data.user module}
\label{\detokenize{flaskapp.data:module-flaskapp.data.user}}\label{\detokenize{flaskapp.data:flaskapp-data-user-module}}\index{module@\spxentry{module}!flaskapp.data.user@\spxentry{flaskapp.data.user}}\index{flaskapp.data.user@\spxentry{flaskapp.data.user}!module@\spxentry{module}}

\subparagraph{User Class}
\label{\detokenize{flaskapp.data:user-class}}
The class to represent a user of this application.
Capable of initializing and accessing the database
specified in the application configuration, as well
as loggin in and out, creating an account, and
modifying the pacemaker parameters of the currently
logged in user.
\index{User (class in flaskapp.data.user)@\spxentry{User}\spxextra{class in flaskapp.data.user}}

\begin{fulllineitems}
\phantomsection\label{\detokenize{flaskapp.data:flaskapp.data.user.User}}\pysiglinewithargsret{\sphinxbfcode{\sphinxupquote{class }}\sphinxcode{\sphinxupquote{flaskapp.data.user.}}\sphinxbfcode{\sphinxupquote{User}}}{\emph{\DUrole{n}{config}}}{}
Bases: \sphinxcode{\sphinxupquote{object}}

This is a class representation of a simple flask app user

The class to represent a user of this application.
Capable of initializing and accessing the database
specified in the application configuration, as well
as loggin in and out, creating an account, and
modifying the pacemaker parameters of the currently
logged in user.
\begin{quote}\begin{description}
\item[{Parameters}] \leavevmode
\sphinxstyleliteralstrong{\sphinxupquote{config}} (class:\sphinxtitleref{configparser.ConfigParser}) \textendash{} A handle to the \sphinxcode{\sphinxupquote{configparser.ConfigParser}} config
object initialized by the main application on startup

\end{description}\end{quote}
\index{create\_account() (flaskapp.data.user.User method)@\spxentry{create\_account()}\spxextra{flaskapp.data.user.User method}}

\begin{fulllineitems}
\phantomsection\label{\detokenize{flaskapp.data:flaskapp.data.user.User.create_account}}\pysiglinewithargsret{\sphinxbfcode{\sphinxupquote{create\_account}}}{\emph{\DUrole{n}{username}}, \emph{\DUrole{n}{password}}}{}
create\_account Creates a new user account

Checks the database to ensure no user with the same username exists,
and that the maximum allowable local\sphinxhyphen{}agents has not been exceeded
(defined in the application.ini). If no conflicts exist, a new user
is created then both inserted into the database and logged in.
\begin{quote}\begin{description}
\item[{Parameters}] \leavevmode\begin{itemize}
\item {} 
\sphinxstyleliteralstrong{\sphinxupquote{username}} (\sphinxstyleliteralemphasis{\sphinxupquote{str}}) \textendash{} The username of the new user being created

\item {} 
\sphinxstyleliteralstrong{\sphinxupquote{password}} (\sphinxstyleliteralemphasis{\sphinxupquote{str}}) \textendash{} The password of the new user being created

\end{itemize}

\item[{Returns}] \leavevmode
True if the account creation was successful, False otherwise

\item[{Return type}] \leavevmode
bool

\end{description}\end{quote}

\end{fulllineitems}

\index{get\_pacemaker\_parameters() (flaskapp.data.user.User method)@\spxentry{get\_pacemaker\_parameters()}\spxextra{flaskapp.data.user.User method}}

\begin{fulllineitems}
\phantomsection\label{\detokenize{flaskapp.data:flaskapp.data.user.User.get_pacemaker_parameters}}\pysiglinewithargsret{\sphinxbfcode{\sphinxupquote{get\_pacemaker\_parameters}}}{}{}
get\_pacemaker\_parameters Returns a dictionary of this users pacemaker parameters
\begin{quote}\begin{description}
\item[{Returns}] \leavevmode
A dictionary of this users pacemaker parameters

\item[{Return type}] \leavevmode
dict

\end{description}\end{quote}

\end{fulllineitems}

\index{get\_username() (flaskapp.data.user.User method)@\spxentry{get\_username()}\spxextra{flaskapp.data.user.User method}}

\begin{fulllineitems}
\phantomsection\label{\detokenize{flaskapp.data:flaskapp.data.user.User.get_username}}\pysiglinewithargsret{\sphinxbfcode{\sphinxupquote{get\_username}}}{}{}
get\_username Returns this users username
\begin{quote}\begin{description}
\item[{Returns}] \leavevmode
This users username

\item[{Return type}] \leavevmode
str

\end{description}\end{quote}

\end{fulllineitems}

\index{is\_loggedin() (flaskapp.data.user.User method)@\spxentry{is\_loggedin()}\spxextra{flaskapp.data.user.User method}}

\begin{fulllineitems}
\phantomsection\label{\detokenize{flaskapp.data:flaskapp.data.user.User.is_loggedin}}\pysiglinewithargsret{\sphinxbfcode{\sphinxupquote{is\_loggedin}}}{}{}
is\_loggedin Checks if this user is logged in
\begin{quote}\begin{description}
\item[{Returns}] \leavevmode
True if the user is logged in, False otherwise

\item[{Return type}] \leavevmode
bool

\end{description}\end{quote}

\end{fulllineitems}

\index{login() (flaskapp.data.user.User method)@\spxentry{login()}\spxextra{flaskapp.data.user.User method}}

\begin{fulllineitems}
\phantomsection\label{\detokenize{flaskapp.data:flaskapp.data.user.User.login}}\pysiglinewithargsret{\sphinxbfcode{\sphinxupquote{login}}}{\emph{\DUrole{n}{username}}, \emph{\DUrole{n}{password}}}{}
login Attempts to log a user in, given their username and password

Searches the database to check if the user with matching username and
password exists. No conflict management (i.e. ensuring only one user
matches that username) is necessary since it is handled on account
creation. If a matching user is found the \sphinxcode{\sphinxupquote{data.user}} is updated
with that users information and the user is logged in.
\begin{quote}\begin{description}
\item[{Parameters}] \leavevmode\begin{itemize}
\item {} 
\sphinxstyleliteralstrong{\sphinxupquote{username}} (\sphinxstyleliteralemphasis{\sphinxupquote{str}}) \textendash{} The username of the user trying to login

\item {} 
\sphinxstyleliteralstrong{\sphinxupquote{password}} (\sphinxstyleliteralemphasis{\sphinxupquote{str}}) \textendash{} The password of the user trying to login

\end{itemize}

\end{description}\end{quote}

\end{fulllineitems}

\index{logout() (flaskapp.data.user.User method)@\spxentry{logout()}\spxextra{flaskapp.data.user.User method}}

\begin{fulllineitems}
\phantomsection\label{\detokenize{flaskapp.data:flaskapp.data.user.User.logout}}\pysiglinewithargsret{\sphinxbfcode{\sphinxupquote{logout}}}{}{}
logout Logs out the currently logged in user

\end{fulllineitems}

\index{num\_parameters (flaskapp.data.user.User attribute)@\spxentry{num\_parameters}\spxextra{flaskapp.data.user.User attribute}}

\begin{fulllineitems}
\phantomsection\label{\detokenize{flaskapp.data:flaskapp.data.user.User.num_parameters}}\pysigline{\sphinxbfcode{\sphinxupquote{num\_parameters}}\sphinxbfcode{\sphinxupquote{ = 8}}}
\end{fulllineitems}

\index{parameters (flaskapp.data.user.User attribute)@\spxentry{parameters}\spxextra{flaskapp.data.user.User attribute}}

\begin{fulllineitems}
\phantomsection\label{\detokenize{flaskapp.data:flaskapp.data.user.User.parameters}}\pysigline{\sphinxbfcode{\sphinxupquote{parameters}}\sphinxbfcode{\sphinxupquote{ = \{\textquotesingle{}Atrial Amplitude\textquotesingle{}: None, \textquotesingle{}Atrial Pulse Width\textquotesingle{}: None, \textquotesingle{}Atrial Refractory Period\textquotesingle{}: None, \textquotesingle{}Lower Rate Limit\textquotesingle{}: None, \textquotesingle{}Upper Rate Limit\textquotesingle{}: None, \textquotesingle{}Ventricular Amplitude\textquotesingle{}: None, \textquotesingle{}Ventricular Pulse Width\textquotesingle{}: None, \textquotesingle{}Ventricular Refractory Period\textquotesingle{}: None\}}}}
\end{fulllineitems}

\index{update\_all\_pacemaker\_parameters() (flaskapp.data.user.User method)@\spxentry{update\_all\_pacemaker\_parameters()}\spxextra{flaskapp.data.user.User method}}

\begin{fulllineitems}
\phantomsection\label{\detokenize{flaskapp.data:flaskapp.data.user.User.update_all_pacemaker_parameters}}\pysiglinewithargsret{\sphinxbfcode{\sphinxupquote{update\_all\_pacemaker\_parameters}}}{\emph{\DUrole{n}{values}}}{}
update\_all\_pacemaker\_parameters Updates all pacemaker parameters

Given a dictionary of pacemaker parameters, in the same order as the values of
the pacemaker parameters dictionary, will update every value of the dictionary.
NOTE: No complete check is done to ensure the validity of the input, it is up
to the method user to ensure the dictionaries correctness.
\begin{quote}\begin{description}
\item[{Parameters}] \leavevmode
\sphinxstyleliteralstrong{\sphinxupquote{values}} (\sphinxstyleliteralemphasis{\sphinxupquote{dict}}) \textendash{} A dictionary of the updated pacemaker parameters

\item[{Returns}] \leavevmode
True if the pacemaker parameters were updated sucessfully, False otherwise

\item[{Return type}] \leavevmode
bool

\end{description}\end{quote}

\end{fulllineitems}

\index{update\_pacemaker\_parameter() (flaskapp.data.user.User method)@\spxentry{update\_pacemaker\_parameter()}\spxextra{flaskapp.data.user.User method}}

\begin{fulllineitems}
\phantomsection\label{\detokenize{flaskapp.data:flaskapp.data.user.User.update_pacemaker_parameter}}\pysiglinewithargsret{\sphinxbfcode{\sphinxupquote{update\_pacemaker\_parameter}}}{\emph{\DUrole{n}{key}}, \emph{\DUrole{n}{value}}}{}
update\_pacemaker\_parameter Updates a single pacemaker parameter

Given a valid key (one already contained in the pacemaker parameters
dictionary), will update the value of that key with the passes in
value.
\begin{quote}\begin{description}
\item[{Parameters}] \leavevmode\begin{itemize}
\item {} 
\sphinxstyleliteralstrong{\sphinxupquote{key}} (\sphinxstyleliteralemphasis{\sphinxupquote{str}}) \textendash{} A key already contained in the pacemaker parameters dictionary

\item {} 
\sphinxstyleliteralstrong{\sphinxupquote{value}} (\sphinxstyleliteralemphasis{\sphinxupquote{int}}) \textendash{} An updated value for the associated key

\end{itemize}

\item[{Returns}] \leavevmode
True if the parameter was sucessfully updated, False otherwise

\item[{Return type}] \leavevmode
bool

\end{description}\end{quote}

\end{fulllineitems}


\end{fulllineitems}



\paragraph{Module contents}
\label{\detokenize{flaskapp.data:module-flaskapp.data}}\label{\detokenize{flaskapp.data:module-contents}}\index{module@\spxentry{module}!flaskapp.data@\spxentry{flaskapp.data}}\index{flaskapp.data@\spxentry{flaskapp.data}!module@\spxentry{module}}

\subsubsection{flaskapp.tests package}
\label{\detokenize{flaskapp.tests:flaskapp-tests-package}}\label{\detokenize{flaskapp.tests::doc}}

\paragraph{Submodules}
\label{\detokenize{flaskapp.tests:submodules}}

\paragraph{flaskapp.tests.test module}
\label{\detokenize{flaskapp.tests:module-flaskapp.tests.test}}\label{\detokenize{flaskapp.tests:flaskapp-tests-test-module}}\index{module@\spxentry{module}!flaskapp.tests.test@\spxentry{flaskapp.tests.test}}\index{flaskapp.tests.test@\spxentry{flaskapp.tests.test}!module@\spxentry{module}}\index{FlaskTestCase (class in flaskapp.tests.test)@\spxentry{FlaskTestCase}\spxextra{class in flaskapp.tests.test}}

\begin{fulllineitems}
\phantomsection\label{\detokenize{flaskapp.tests:flaskapp.tests.test.FlaskTestCase}}\pysiglinewithargsret{\sphinxbfcode{\sphinxupquote{class }}\sphinxcode{\sphinxupquote{flaskapp.tests.test.}}\sphinxbfcode{\sphinxupquote{FlaskTestCase}}}{\emph{\DUrole{n}{methodName}\DUrole{o}{=}\DUrole{default_value}{\textquotesingle{}runTest\textquotesingle{}}}}{}
Bases: \sphinxcode{\sphinxupquote{unittest.case.TestCase}}
\index{test\_login() (flaskapp.tests.test.FlaskTestCase method)@\spxentry{test\_login()}\spxextra{flaskapp.tests.test.FlaskTestCase method}}

\begin{fulllineitems}
\phantomsection\label{\detokenize{flaskapp.tests:flaskapp.tests.test.FlaskTestCase.test_login}}\pysiglinewithargsret{\sphinxbfcode{\sphinxupquote{test\_login}}}{}{}
\end{fulllineitems}

\index{test\_login\_correct() (flaskapp.tests.test.FlaskTestCase method)@\spxentry{test\_login\_correct()}\spxextra{flaskapp.tests.test.FlaskTestCase method}}

\begin{fulllineitems}
\phantomsection\label{\detokenize{flaskapp.tests:flaskapp.tests.test.FlaskTestCase.test_login_correct}}\pysiglinewithargsret{\sphinxbfcode{\sphinxupquote{test\_login\_correct}}}{}{}
\end{fulllineitems}

\index{test\_login\_incorrect() (flaskapp.tests.test.FlaskTestCase method)@\spxentry{test\_login\_incorrect()}\spxextra{flaskapp.tests.test.FlaskTestCase method}}

\begin{fulllineitems}
\phantomsection\label{\detokenize{flaskapp.tests:flaskapp.tests.test.FlaskTestCase.test_login_incorrect}}\pysiglinewithargsret{\sphinxbfcode{\sphinxupquote{test\_login\_incorrect}}}{}{}
\end{fulllineitems}

\index{test\_login\_loads() (flaskapp.tests.test.FlaskTestCase method)@\spxentry{test\_login\_loads()}\spxextra{flaskapp.tests.test.FlaskTestCase method}}

\begin{fulllineitems}
\phantomsection\label{\detokenize{flaskapp.tests:flaskapp.tests.test.FlaskTestCase.test_login_loads}}\pysiglinewithargsret{\sphinxbfcode{\sphinxupquote{test\_login\_loads}}}{}{}
\end{fulllineitems}

\index{test\_logout\_works() (flaskapp.tests.test.FlaskTestCase method)@\spxentry{test\_logout\_works()}\spxextra{flaskapp.tests.test.FlaskTestCase method}}

\begin{fulllineitems}
\phantomsection\label{\detokenize{flaskapp.tests:flaskapp.tests.test.FlaskTestCase.test_logout_works}}\pysiglinewithargsret{\sphinxbfcode{\sphinxupquote{test\_logout\_works}}}{}{}
\end{fulllineitems}

\index{test\_user\_blocked() (flaskapp.tests.test.FlaskTestCase method)@\spxentry{test\_user\_blocked()}\spxextra{flaskapp.tests.test.FlaskTestCase method}}

\begin{fulllineitems}
\phantomsection\label{\detokenize{flaskapp.tests:flaskapp.tests.test.FlaskTestCase.test_user_blocked}}\pysiglinewithargsret{\sphinxbfcode{\sphinxupquote{test\_user\_blocked}}}{}{}
\end{fulllineitems}


\end{fulllineitems}



\paragraph{Module contents}
\label{\detokenize{flaskapp.tests:module-flaskapp.tests}}\label{\detokenize{flaskapp.tests:module-contents}}\index{module@\spxentry{module}!flaskapp.tests@\spxentry{flaskapp.tests}}\index{flaskapp.tests@\spxentry{flaskapp.tests}!module@\spxentry{module}}

\subsection{Submodules}
\label{\detokenize{flaskapp:submodules}}

\subsection{flaskapp.app module}
\label{\detokenize{flaskapp:module-flaskapp.app}}\label{\detokenize{flaskapp:flaskapp-app-module}}\index{module@\spxentry{module}!flaskapp.app@\spxentry{flaskapp.app}}\index{flaskapp.app@\spxentry{flaskapp.app}!module@\spxentry{module}}

\subsubsection{Main Application}
\label{\detokenize{flaskapp:main-application}}
The main implementation of the DCM flaskapp.
Handles rendering off all the endpoints as well
as communication between the frontend and
backend.
\index{home() (in module flaskapp.app)@\spxentry{home()}\spxextra{in module flaskapp.app}}

\begin{fulllineitems}
\phantomsection\label{\detokenize{flaskapp:flaskapp.app.home}}\pysiglinewithargsret{\sphinxcode{\sphinxupquote{flaskapp.app.}}\sphinxbfcode{\sphinxupquote{home}}}{}{}
home The route to the homepage of the flask application

Renders the homepage of the flask app and manages post requests.
Supported post request are ‘login’ and ‘account creation’ requests,
which when completed successfully will redirect the user to their
user specific page. Any failed post request will result in an error
message flashed to the screen.
\begin{quote}\begin{description}
\item[{Returns}] \leavevmode
The render template used by the homepage, in this case the ‘index.html’ template

\item[{Return type}] \leavevmode
\sphinxcode{\sphinxupquote{flask.render\_tmeplate}}

\end{description}\end{quote}

\end{fulllineitems}

\index{logout() (in module flaskapp.app)@\spxentry{logout()}\spxextra{in module flaskapp.app}}

\begin{fulllineitems}
\phantomsection\label{\detokenize{flaskapp:flaskapp.app.logout}}\pysiglinewithargsret{\sphinxcode{\sphinxupquote{flaskapp.app.}}\sphinxbfcode{\sphinxupquote{logout}}}{}{}
logout The route to the logout page of the flask application

This page acts as an intermediary between a user page that requires login
and the homepage of this application. This page logs the user out and
immediately redirects to the homepage, flashing a logout message after
the redirect.
\begin{quote}\begin{description}
\item[{Returns}] \leavevmode
A redirect to the homepage of the application

\item[{Return type}] \leavevmode
\sphinxcode{\sphinxupquote{flask.redirect}}

\end{description}\end{quote}

\end{fulllineitems}

\index{open\_browser() (in module flaskapp.app)@\spxentry{open\_browser()}\spxextra{in module flaskapp.app}}

\begin{fulllineitems}
\phantomsection\label{\detokenize{flaskapp:flaskapp.app.open_browser}}\pysiglinewithargsret{\sphinxcode{\sphinxupquote{flaskapp.app.}}\sphinxbfcode{\sphinxupquote{open\_browser}}}{}{}
open\_browser Opens the flask app automatically in the web brower.

So the user doesn’t have to memorize the url and port the application
is being hosted on.

\end{fulllineitems}

\index{user\_connect() (in module flaskapp.app)@\spxentry{user\_connect()}\spxextra{in module flaskapp.app}}

\begin{fulllineitems}
\phantomsection\label{\detokenize{flaskapp:flaskapp.app.user_connect}}\pysiglinewithargsret{\sphinxcode{\sphinxupquote{flaskapp.app.}}\sphinxbfcode{\sphinxupquote{user\_connect}}}{}{}
user\_connect The route to the user connect page of the flask application

Renders the user connect page of the flask app. This page allows the user to
view the status of the pacemaker in different pacing modes. Pacing modes can
be changed via a selection box at the top of the page. NOTE: This page will
correctly change between different pacing mode states, though these states
and their corresponding rendered templates are black as no serial communication
with the pacemaker has been implemented. As a result changing modes will have little
visible effect on the rendered content of the application.
\begin{quote}\begin{description}
\item[{Returns}] \leavevmode
The render template used by the user connect page, in this case the ‘user\_connect.html’ template

\item[{Return type}] \leavevmode
\sphinxcode{\sphinxupquote{flask.render\_tmeplate}}

\end{description}\end{quote}

\end{fulllineitems}

\index{user\_page() (in module flaskapp.app)@\spxentry{user\_page()}\spxextra{in module flaskapp.app}}

\begin{fulllineitems}
\phantomsection\label{\detokenize{flaskapp:flaskapp.app.user_page}}\pysiglinewithargsret{\sphinxcode{\sphinxupquote{flaskapp.app.}}\sphinxbfcode{\sphinxupquote{user\_page}}}{}{}
user\_page The route to the user specific page of the flask application

Renders the user specific page of the flask app. This page acts as an intermediary
between the ‘login’ state and any other states which require login, such as the
user parameters and pacing mode specific states.
\begin{quote}\begin{description}
\item[{Returns}] \leavevmode
The render template used by the user specific page, in this case the ‘user.html’ template

\item[{Return type}] \leavevmode
\sphinxcode{\sphinxupquote{flask.render\_tmeplate}}

\end{description}\end{quote}

\end{fulllineitems}

\index{user\_parameters() (in module flaskapp.app)@\spxentry{user\_parameters()}\spxextra{in module flaskapp.app}}

\begin{fulllineitems}
\phantomsection\label{\detokenize{flaskapp:flaskapp.app.user_parameters}}\pysiglinewithargsret{\sphinxcode{\sphinxupquote{flaskapp.app.}}\sphinxbfcode{\sphinxupquote{user\_parameters}}}{}{}
user\_parameters The route to the user parameters page of the flask application

Renders the user parameters page of the flask app. This page allows the user to
view and modify their pacemaker parameters through a submittable form. Users can
modify some or all parameters and submit the changes through a post request. Changes
will be made immediately in the flask app and database, and the user parameters page
updated with these changed values.
\begin{quote}\begin{description}
\item[{Returns}] \leavevmode
The render template used by the user parameters page, in this case the ‘user\_parameters.html’ template

\item[{Return type}] \leavevmode
\sphinxcode{\sphinxupquote{flask.render\_tmeplate}}

\end{description}\end{quote}

\end{fulllineitems}



\subsection{Module contents}
\label{\detokenize{flaskapp:module-flaskapp}}\label{\detokenize{flaskapp:module-contents}}\index{module@\spxentry{module}!flaskapp@\spxentry{flaskapp}}\index{flaskapp@\spxentry{flaskapp}!module@\spxentry{module}}

\chapter{Indices and tables}
\label{\detokenize{index:indices-and-tables}}\begin{itemize}
\item {} 
\DUrole{xref,std,std-ref}{genindex}

\item {} 
\DUrole{xref,std,std-ref}{modindex}

\item {} 
\DUrole{xref,std,std-ref}{search}

\end{itemize}


\renewcommand{\indexname}{Python Module Index}
\begin{sphinxtheindex}
\let\bigletter\sphinxstyleindexlettergroup
\bigletter{f}
\item\relax\sphinxstyleindexentry{flaskapp}\sphinxstyleindexpageref{flaskapp:\detokenize{module-flaskapp}}
\item\relax\sphinxstyleindexentry{flaskapp.app}\sphinxstyleindexpageref{flaskapp:\detokenize{module-flaskapp.app}}
\item\relax\sphinxstyleindexentry{flaskapp.config}\sphinxstyleindexpageref{flaskapp.config:\detokenize{module-flaskapp.config}}
\item\relax\sphinxstyleindexentry{flaskapp.config.config\_manager}\sphinxstyleindexpageref{flaskapp.config:\detokenize{module-flaskapp.config.config_manager}}
\item\relax\sphinxstyleindexentry{flaskapp.config.decorators}\sphinxstyleindexpageref{flaskapp.config:\detokenize{module-flaskapp.config.decorators}}
\item\relax\sphinxstyleindexentry{flaskapp.data}\sphinxstyleindexpageref{flaskapp.data:\detokenize{module-flaskapp.data}}
\item\relax\sphinxstyleindexentry{flaskapp.data.database}\sphinxstyleindexpageref{flaskapp.data:\detokenize{module-flaskapp.data.database}}
\item\relax\sphinxstyleindexentry{flaskapp.data.user}\sphinxstyleindexpageref{flaskapp.data:\detokenize{module-flaskapp.data.user}}
\item\relax\sphinxstyleindexentry{flaskapp.tests}\sphinxstyleindexpageref{flaskapp.tests:\detokenize{module-flaskapp.tests}}
\item\relax\sphinxstyleindexentry{flaskapp.tests.test}\sphinxstyleindexpageref{flaskapp.tests:\detokenize{module-flaskapp.tests.test}}
\end{sphinxtheindex}

\renewcommand{\indexname}{Index}
\printindex
\end{document}