%% Generated by Sphinx.
\def\sphinxdocclass{report}
\documentclass[letterpaper,10pt,english]{sphinxmanual}
\ifdefined\pdfpxdimen
   \let\sphinxpxdimen\pdfpxdimen\else\newdimen\sphinxpxdimen
\fi \sphinxpxdimen=.75bp\relax

\PassOptionsToPackage{warn}{textcomp}
\usepackage[utf8]{inputenc}
\ifdefined\DeclareUnicodeCharacter
% support both utf8 and utf8x syntaxes
  \ifdefined\DeclareUnicodeCharacterAsOptional
    \def\sphinxDUC#1{\DeclareUnicodeCharacter{"#1}}
  \else
    \let\sphinxDUC\DeclareUnicodeCharacter
  \fi
  \sphinxDUC{00A0}{\nobreakspace}
  \sphinxDUC{2500}{\sphinxunichar{2500}}
  \sphinxDUC{2502}{\sphinxunichar{2502}}
  \sphinxDUC{2514}{\sphinxunichar{2514}}
  \sphinxDUC{251C}{\sphinxunichar{251C}}
  \sphinxDUC{2572}{\textbackslash}
\fi
\usepackage{cmap}
\usepackage[T1]{fontenc}
\usepackage{amsmath,amssymb,amstext}
\usepackage{babel}



\usepackage{times}
\expandafter\ifx\csname T@LGR\endcsname\relax
\else
% LGR was declared as font encoding
  \substitutefont{LGR}{\rmdefault}{cmr}
  \substitutefont{LGR}{\sfdefault}{cmss}
  \substitutefont{LGR}{\ttdefault}{cmtt}
\fi
\expandafter\ifx\csname T@X2\endcsname\relax
  \expandafter\ifx\csname T@T2A\endcsname\relax
  \else
  % T2A was declared as font encoding
    \substitutefont{T2A}{\rmdefault}{cmr}
    \substitutefont{T2A}{\sfdefault}{cmss}
    \substitutefont{T2A}{\ttdefault}{cmtt}
  \fi
\else
% X2 was declared as font encoding
  \substitutefont{X2}{\rmdefault}{cmr}
  \substitutefont{X2}{\sfdefault}{cmss}
  \substitutefont{X2}{\ttdefault}{cmtt}
\fi


\usepackage[Bjarne]{fncychap}
\usepackage{sphinx}

\fvset{fontsize=\small}
\usepackage{geometry}


% Include hyperref last.
\usepackage{hyperref}
% Fix anchor placement for figures with captions.
\usepackage{hypcap}% it must be loaded after hyperref.
% Set up styles of URL: it should be placed after hyperref.
\urlstyle{same}

\addto\captionsenglish{\renewcommand{\contentsname}{Contents:}}

\usepackage{sphinxmessages}
\setcounter{tocdepth}{1}



\title{Pacemaker DCM (Group 10)}
\date{Nov 26, 2020}
\release{}
\author{Graham Power, Hamza Ashraf}
\newcommand{\sphinxlogo}{\vbox{}}
\renewcommand{\releasename}{}
\makeindex
\begin{document}

\pagestyle{empty}
\sphinxmaketitle
\pagestyle{plain}
\sphinxtableofcontents
\pagestyle{normal}
\phantomsection\label{\detokenize{index::doc}}



\chapter{FlaskApp App}
\label{\detokenize{index:module-app}}\label{\detokenize{index:flaskapp-app}}\index{module@\spxentry{module}!app@\spxentry{app}}\index{app@\spxentry{app}!module@\spxentry{module}}

\section{Main Application}
\label{\detokenize{index:main-application}}
The main implementation of the DCM flaskapp.
Handles rendering off all the endpoints as well
as communication between the frontend and
backend.
\index{home() (in module app)@\spxentry{home()}\spxextra{in module app}}

\begin{fulllineitems}
\phantomsection\label{\detokenize{index:app.home}}\pysiglinewithargsret{\sphinxcode{\sphinxupquote{app.}}\sphinxbfcode{\sphinxupquote{home}}}{}{}
home The route to the homepage of the flask application

Renders the homepage of the flask app and manages post requests.
Supported post request are ‘login’ and ‘account creation’ requests,
which when completed successfully will redirect the user to their
user specific page. Any failed post request will result in an error
message flashed to the screen.
\begin{quote}\begin{description}
\item[{Returns}] \leavevmode
The render template used by the homepage, in this case the ‘index.html’ template

\item[{Return type}] \leavevmode
\sphinxcode{\sphinxupquote{flask.render\_tmeplate}}

\end{description}\end{quote}

\end{fulllineitems}

\index{logout() (in module app)@\spxentry{logout()}\spxextra{in module app}}

\begin{fulllineitems}
\phantomsection\label{\detokenize{index:app.logout}}\pysiglinewithargsret{\sphinxcode{\sphinxupquote{app.}}\sphinxbfcode{\sphinxupquote{logout}}}{}{}
logout The route to the logout page of the flask application

This page acts as an intermediary between a user page that requires login
and the homepage of this application. This page logs the user out and
immediately redirects to the homepage, flashing a logout message after
the redirect.
\begin{quote}\begin{description}
\item[{Returns}] \leavevmode
A redirect to the homepage of the application

\item[{Return type}] \leavevmode
\sphinxcode{\sphinxupquote{flask.redirect}}

\end{description}\end{quote}

\end{fulllineitems}

\index{open\_browser() (in module app)@\spxentry{open\_browser()}\spxextra{in module app}}

\begin{fulllineitems}
\phantomsection\label{\detokenize{index:app.open_browser}}\pysiglinewithargsret{\sphinxcode{\sphinxupquote{app.}}\sphinxbfcode{\sphinxupquote{open\_browser}}}{}{}
open\_browser Opens the flask app automatically in the web brower.

So the user doesn’t have to memorize the url and port the application
is being hosted on.

\end{fulllineitems}

\index{random() (in module app)@\spxentry{random()}\spxextra{in module app}}

\begin{fulllineitems}
\phantomsection\label{\detokenize{index:app.random}}\pysiglinewithargsret{\sphinxcode{\sphinxupquote{app.}}\sphinxbfcode{\sphinxupquote{random}}}{}{{ $\rightarrow$ x in the interval {[}0, 1).}}
\end{fulllineitems}

\index{user\_connect() (in module app)@\spxentry{user\_connect()}\spxextra{in module app}}

\begin{fulllineitems}
\phantomsection\label{\detokenize{index:app.user_connect}}\pysiglinewithargsret{\sphinxcode{\sphinxupquote{app.}}\sphinxbfcode{\sphinxupquote{user\_connect}}}{}{}
user\_connect The route to the user connect page of the flask application

Renders the user connect page of the flask app. This page allows the user to
view the status of the pacemaker in different pacing modes. Pacing modes can
be changed via a selection box at the top of the page. NOTE: This page will
correctly change between different pacing mode states, though these states
and their corresponding rendered templates are black as no serial communication
with the pacemaker has been implemented. As a result changing modes will have little
visible effect on the rendered content of the application.
\begin{quote}\begin{description}
\item[{Returns}] \leavevmode
The render template used by the user connect page, in this case the ‘user\_connect.html’ template

\item[{Return type}] \leavevmode
\sphinxcode{\sphinxupquote{flask.render\_tmeplate}}

\end{description}\end{quote}

\end{fulllineitems}

\index{user\_data() (in module app)@\spxentry{user\_data()}\spxextra{in module app}}

\begin{fulllineitems}
\phantomsection\label{\detokenize{index:app.user_data}}\pysiglinewithargsret{\sphinxcode{\sphinxupquote{app.}}\sphinxbfcode{\sphinxupquote{user\_data}}}{}{}
user\_data The route to the data request uri for the egram

Will return a json response containing updated point values to
be added to the live graph.
\begin{quote}\begin{description}
\item[{Returns}] \leavevmode
A json response, containing updated point values

\item[{Return type}] \leavevmode
json

\end{description}\end{quote}

\end{fulllineitems}

\index{user\_egram() (in module app)@\spxentry{user\_egram()}\spxextra{in module app}}

\begin{fulllineitems}
\phantomsection\label{\detokenize{index:app.user_egram}}\pysiglinewithargsret{\sphinxcode{\sphinxupquote{app.}}\sphinxbfcode{\sphinxupquote{user\_egram}}}{}{}
user\_egram The route to the user history page of the flask application

Renders the user egram page of the flask app. This page allows the user to
view a live graph (or egram) of the pacemakers atrium and ventrical chambers.
NOTE: The graph displayed is an adaptable graph and will take up to one minute
to adjust both the graphs domain and range values.
\begin{quote}\begin{description}
\item[{Returns}] \leavevmode
The render template used by the user connect page, in this case the ‘user\_egram.html’ template

\item[{Return type}] \leavevmode
\sphinxcode{\sphinxupquote{flask.render\_tmeplate}}

\end{description}\end{quote}

\end{fulllineitems}

\index{user\_history() (in module app)@\spxentry{user\_history()}\spxextra{in module app}}

\begin{fulllineitems}
\phantomsection\label{\detokenize{index:app.user_history}}\pysiglinewithargsret{\sphinxcode{\sphinxupquote{app.}}\sphinxbfcode{\sphinxupquote{user\_history}}}{}{}
user\_history The route to the user history page of the flask application

Renders the user history page of the flask app. This page allows the user to
view the entire history of their pacemaker parameters in a convenient table.
This table can be exported to a csv file if the user wishes, which will appear
in the downloads file.
\begin{quote}\begin{description}
\item[{Returns}] \leavevmode
The render template used by the user connect page, in this case the ‘user\_history.html’ template

\item[{Return type}] \leavevmode
\sphinxcode{\sphinxupquote{flask.render\_tmeplate}}

\end{description}\end{quote}

\end{fulllineitems}

\index{user\_page() (in module app)@\spxentry{user\_page()}\spxextra{in module app}}

\begin{fulllineitems}
\phantomsection\label{\detokenize{index:app.user_page}}\pysiglinewithargsret{\sphinxcode{\sphinxupquote{app.}}\sphinxbfcode{\sphinxupquote{user\_page}}}{}{}
user\_page The route to the user specific page of the flask application

Renders the user specific page of the flask app. This page acts as an intermediary
between the ‘login’ state and any other states which require login, such as the
user parameters and pacing mode specific states.
\begin{quote}\begin{description}
\item[{Returns}] \leavevmode
The render template used by the user specific page, in this case the ‘user.html’ template

\item[{Return type}] \leavevmode
\sphinxcode{\sphinxupquote{flask.render\_tmeplate}}

\end{description}\end{quote}

\end{fulllineitems}

\index{user\_parameters() (in module app)@\spxentry{user\_parameters()}\spxextra{in module app}}

\begin{fulllineitems}
\phantomsection\label{\detokenize{index:app.user_parameters}}\pysiglinewithargsret{\sphinxcode{\sphinxupquote{app.}}\sphinxbfcode{\sphinxupquote{user\_parameters}}}{}{}
user\_parameters The route to the user parameters page of the flask application

Renders the user parameters page of the flask app. This page allows the user to
view and modify their pacemaker parameters through a submittable form. Users can
modify some or all parameters and submit the changes through a post request. Changes
will be made immediately in the flask app and database, and the user parameters page
updated with these changed values.
\begin{quote}\begin{description}
\item[{Returns}] \leavevmode
The render template used by the user parameters page, in this case the ‘user\_parameters.html’ template

\item[{Return type}] \leavevmode
\sphinxcode{\sphinxupquote{flask.render\_tmeplate}}

\end{description}\end{quote}

\end{fulllineitems}



\chapter{Config Config\_Manager}
\label{\detokenize{index:module-config.config_manager}}\label{\detokenize{index:config-config-manager}}\index{module@\spxentry{module}!config.config\_manager@\spxentry{config.config\_manager}}\index{config.config\_manager@\spxentry{config.config\_manager}!module@\spxentry{module}}

\section{Config Manager}
\label{\detokenize{index:config-manager}}
A collection of configuration setup
classes, to initialize the config
variables and setup application
logging as well as allow access to
both the config variables and logging
function directly form anwhere within
the application.
\index{Config (class in config.config\_manager)@\spxentry{Config}\spxextra{class in config.config\_manager}}

\begin{fulllineitems}
\phantomsection\label{\detokenize{index:config.config_manager.Config}}\pysigline{\sphinxbfcode{\sphinxupquote{class }}\sphinxcode{\sphinxupquote{config.config\_manager.}}\sphinxbfcode{\sphinxupquote{Config}}}
The configuration class, holds all the configuration information

Manages the configuration and provides access to the config through a
singleton implementation. Its get method can be called from anywhere
within the application the Config class is imported.
\begin{quote}\begin{description}
\item[{Raises}] \leavevmode\begin{itemize}
\item {} 
\sphinxstyleliteralstrong{\sphinxupquote{Exception}} \textendash{} This class is a singleton!

\item {} 
\sphinxstyleliteralstrong{\sphinxupquote{Exception}} \textendash{} Configuration can only be read once!

\item {} 
\sphinxstyleliteralstrong{\sphinxupquote{Exception}} \textendash{} Configutation has not been read, read\_config() must be called first!

\end{itemize}

\end{description}\end{quote}
\index{get() (config.config\_manager.Config method)@\spxentry{get()}\spxextra{config.config\_manager.Config method}}

\begin{fulllineitems}
\phantomsection\label{\detokenize{index:config.config_manager.Config.get}}\pysiglinewithargsret{\sphinxbfcode{\sphinxupquote{get}}}{\emph{\DUrole{n}{section}}, \emph{\DUrole{n}{variable}}}{}
get Gets a variable from the configuration

Given a configuration section and variable name within that section
will return the value of the variable as a string
\begin{quote}\begin{description}
\item[{Parameters}] \leavevmode\begin{itemize}
\item {} 
\sphinxstyleliteralstrong{\sphinxupquote{section}} (\sphinxstyleliteralemphasis{\sphinxupquote{str}}) \textendash{} The section in the configuration file to find the variable in

\item {} 
\sphinxstyleliteralstrong{\sphinxupquote{variable}} (\sphinxstyleliteralemphasis{\sphinxupquote{str}}) \textendash{} The variable name to read from the section

\end{itemize}

\item[{Raises}] \leavevmode
\sphinxstyleliteralstrong{\sphinxupquote{Exception}} \textendash{} Configutation has not been read, read\_config() must be called first!

\item[{Returns}] \leavevmode
The string representation of the variable

\item[{Return type}] \leavevmode
str

\end{description}\end{quote}

\end{fulllineitems}

\index{getInstance() (config.config\_manager.Config static method)@\spxentry{getInstance()}\spxextra{config.config\_manager.Config static method}}

\begin{fulllineitems}
\phantomsection\label{\detokenize{index:config.config_manager.Config.getInstance}}\pysiglinewithargsret{\sphinxbfcode{\sphinxupquote{static }}\sphinxbfcode{\sphinxupquote{getInstance}}}{}{}
getInstance Gets the current instance of Config

If no instance exists, it will create the initial instance
before returning it.
\begin{quote}\begin{description}
\item[{Returns}] \leavevmode
A singleton instance of the Config class

\item[{Return type}] \leavevmode
\sphinxcode{\sphinxupquote{config\_manager.Config}}

\end{description}\end{quote}

\end{fulllineitems}

\index{get\_all() (config.config\_manager.Config method)@\spxentry{get\_all()}\spxextra{config.config\_manager.Config method}}

\begin{fulllineitems}
\phantomsection\label{\detokenize{index:config.config_manager.Config.get_all}}\pysiglinewithargsret{\sphinxbfcode{\sphinxupquote{get\_all}}}{\emph{\DUrole{n}{section}}}{}
get\_all Gets all variable and values within a section

Given a configuration section from the configuration files
will return a dictionary containing all variable names and
values from within that section. All varaible names and values
are stored in their string representation within the dictionary.
\begin{quote}\begin{description}
\item[{Parameters}] \leavevmode
\sphinxstyleliteralstrong{\sphinxupquote{section}} (\sphinxstyleliteralemphasis{\sphinxupquote{str}}) \textendash{} The section of the configuration file to read

\item[{Raises}] \leavevmode
\sphinxstyleliteralstrong{\sphinxupquote{Exception}} \textendash{} Configutation has not been read, read\_config() must be called first!

\item[{Returns}] \leavevmode
A dictionary conatianing variable name, value pairs

\item[{Return type}] \leavevmode
dict

\end{description}\end{quote}

\end{fulllineitems}

\index{getboolean() (config.config\_manager.Config method)@\spxentry{getboolean()}\spxextra{config.config\_manager.Config method}}

\begin{fulllineitems}
\phantomsection\label{\detokenize{index:config.config_manager.Config.getboolean}}\pysiglinewithargsret{\sphinxbfcode{\sphinxupquote{getboolean}}}{\emph{\DUrole{n}{section}}, \emph{\DUrole{n}{variable}}}{}
getboolean Gets a variable from the configuration

Given a configuration section and variable name within that section
will return the value of the variable as a boolean. The value in the
configuration file can only be in the form {[}yes/no, true/false, on/off, 1/0{]},
for any other values this method will throw an error
\begin{quote}\begin{description}
\item[{Parameters}] \leavevmode\begin{itemize}
\item {} 
\sphinxstyleliteralstrong{\sphinxupquote{section}} (\sphinxstyleliteralemphasis{\sphinxupquote{str}}) \textendash{} The section of the configuration file to find the variable in

\item {} 
\sphinxstyleliteralstrong{\sphinxupquote{variable}} (\sphinxstyleliteralemphasis{\sphinxupquote{str}}) \textendash{} The variable to read form the section

\end{itemize}

\item[{Raises}] \leavevmode
\sphinxstyleliteralstrong{\sphinxupquote{Exception}} \textendash{} Configuration has not been rean, read\_config() must be called first!

\item[{Returns}] \leavevmode
The boolean representation of the variable

\item[{Return type}] \leavevmode
bool

\end{description}\end{quote}

\end{fulllineitems}

\index{read\_config() (config.config\_manager.Config method)@\spxentry{read\_config()}\spxextra{config.config\_manager.Config method}}

\begin{fulllineitems}
\phantomsection\label{\detokenize{index:config.config_manager.Config.read_config}}\pysiglinewithargsret{\sphinxbfcode{\sphinxupquote{read\_config}}}{\emph{\DUrole{n}{cfg\_files}\DUrole{o}{=}\DUrole{default_value}{{[}\textquotesingle{}D:\textbackslash{}\textbackslash{}School\textbackslash{}\textbackslash{}Year 3\textbackslash{}\textbackslash{}Semester 1\textbackslash{}\textbackslash{}3K04\textbackslash{}\textbackslash{}PacemakerProject\textbackslash{}\textbackslash{}DCM\textbackslash{}\textbackslash{}flaskapp\textbackslash{}\textbackslash{}config\textbackslash{}\textbackslash{}application.ini\textquotesingle{}{]}}}}{}
read\_config Reads the configuration files

Called on instance initialization. Will search the config
directory for all .ini files and read their variable values
into the Config Manager. Raises an exception if called more
than once, to ensure variables are not changed mid execution.
\begin{quote}\begin{description}
\item[{Parameters}] \leavevmode
\sphinxstyleliteralstrong{\sphinxupquote{cfg\_files}} (\sphinxstyleliteralemphasis{\sphinxupquote{list}}\sphinxstyleliteralemphasis{\sphinxupquote{, }}\sphinxstyleliteralemphasis{\sphinxupquote{optional}}) \textendash{} A list of additional ini files to read, application.ini is added automatically, defaults to {[}{]}

\item[{Raises}] \leavevmode
\sphinxstyleliteralstrong{\sphinxupquote{Exception}} \textendash{} Configuration can only be read once!

\end{description}\end{quote}

\end{fulllineitems}


\end{fulllineitems}

\index{Logger (class in config.config\_manager)@\spxentry{Logger}\spxextra{class in config.config\_manager}}

\begin{fulllineitems}
\phantomsection\label{\detokenize{index:config.config_manager.Logger}}\pysigline{\sphinxbfcode{\sphinxupquote{class }}\sphinxcode{\sphinxupquote{config.config\_manager.}}\sphinxbfcode{\sphinxupquote{Logger}}}
The Logger class, holds all logging information

This class can initialize a logger through a singleton
implementation. Its method can be called from anywhere
within the application where the Logger class is imported.
\begin{quote}\begin{description}
\item[{Raises}] \leavevmode\begin{itemize}
\item {} 
\sphinxstyleliteralstrong{\sphinxupquote{Exception}} \textendash{} This class is a singleton!

\item {} 
\sphinxstyleliteralstrong{\sphinxupquote{Exception}} \textendash{} Logger can only be started once!

\end{itemize}

\end{description}\end{quote}
\index{getInstance() (config.config\_manager.Logger static method)@\spxentry{getInstance()}\spxextra{config.config\_manager.Logger static method}}

\begin{fulllineitems}
\phantomsection\label{\detokenize{index:config.config_manager.Logger.getInstance}}\pysiglinewithargsret{\sphinxbfcode{\sphinxupquote{static }}\sphinxbfcode{\sphinxupquote{getInstance}}}{}{}
getInstance Get the current instance of Logger

If no instance exists it will create the initial
instance before returning it
\begin{quote}\begin{description}
\item[{Returns}] \leavevmode
A singleton instance of the Logger class

\item[{Return type}] \leavevmode
\sphinxcode{\sphinxupquote{config\_manager.Logger}}

\end{description}\end{quote}

\end{fulllineitems}

\index{log() (config.config\_manager.Logger method)@\spxentry{log()}\spxextra{config.config\_manager.Logger method}}

\begin{fulllineitems}
\phantomsection\label{\detokenize{index:config.config_manager.Logger.log}}\pysiglinewithargsret{\sphinxbfcode{\sphinxupquote{log}}}{\emph{\DUrole{n}{level}}, \emph{\DUrole{n}{msg}}}{}
log Will log a message

Given a message and the level of that message, this
method will log the message to all the Loggers handlers
(i.e. file and terminal) only if the log level is equal
to or higher than the Logger’s log level defined in the
applications configuration.
\begin{quote}\begin{description}
\item[{Parameters}] \leavevmode\begin{itemize}
\item {} 
\sphinxstyleliteralstrong{\sphinxupquote{level}} (\sphinxstyleliteralemphasis{\sphinxupquote{str}}) \textendash{} The level of the message being logged can take the values: DEBUG, INFO, WARN, ERROR, CRITICAL

\item {} 
\sphinxstyleliteralstrong{\sphinxupquote{msg}} (\sphinxstyleliteralemphasis{\sphinxupquote{str}}) \textendash{} The message to be logged by the Logger

\end{itemize}

\end{description}\end{quote}

\end{fulllineitems}

\index{start\_logger() (config.config\_manager.Logger method)@\spxentry{start\_logger()}\spxextra{config.config\_manager.Logger method}}

\begin{fulllineitems}
\phantomsection\label{\detokenize{index:config.config_manager.Logger.start_logger}}\pysiglinewithargsret{\sphinxbfcode{\sphinxupquote{start\_logger}}}{\emph{\DUrole{n}{config}}}{}
start\_logger Starts the logger

This method must be called by the main block of the application.
It will create the log file and start the logging process both to
the log file and the terminal
\begin{quote}\begin{description}
\item[{Parameters}] \leavevmode
\sphinxstyleliteralstrong{\sphinxupquote{config}} (\sphinxcode{\sphinxupquote{config\_manager.Config}}) \textendash{} The Instance of Config, which must be created and initialized before the Logger is started

\item[{Raises}] \leavevmode
\sphinxstyleliteralstrong{\sphinxupquote{Exception}} \textendash{} Logger can only be started once!

\end{description}\end{quote}

\end{fulllineitems}


\end{fulllineitems}



\chapter{Config Decorators}
\label{\detokenize{index:module-config.decorators}}\label{\detokenize{index:config-decorators}}\index{module@\spxentry{module}!config.decorators@\spxentry{config.decorators}}\index{config.decorators@\spxentry{config.decorators}!module@\spxentry{module}}

\section{Decorators Library}
\label{\detokenize{index:decorators-library}}
A collection of decorators used by the main
app, created to avoid the unnecessary
reproduction of code.
\index{login\_required() (in module config.decorators)@\spxentry{login\_required()}\spxextra{in module config.decorators}}

\begin{fulllineitems}
\phantomsection\label{\detokenize{index:config.decorators.login_required}}\pysiglinewithargsret{\sphinxcode{\sphinxupquote{config.decorators.}}\sphinxbfcode{\sphinxupquote{login\_required}}}{\emph{\DUrole{n}{f}}}{}
login\_required Ensures that a user is logged in before granting access to user restricted pages

Will check if a user is logged in and only allow a redirect to the user restricted page if there is
a user logged in. If no user is logged in it will chenge the redirect to the home page and flash a
login required messege, letting the user know they must login before accessing that endpoint.
NOTE: This function should never be called as a function, only used as a decorator above functions
who require its functionality to be implemented on entry. Always use a decorator (i.e. @login\_required)
to invoke this function.
\begin{quote}\begin{description}
\item[{Parameters}] \leavevmode
\sphinxstyleliteralstrong{\sphinxupquote{f}} (\sphinxstyleliteralemphasis{\sphinxupquote{function}}) \textendash{} The function being decorated. This will be automatically filled when using the @ tag

\item[{Returns}] \leavevmode
The wrap function, whose contents are the input function, modified to add the functionality of this decorator

\item[{Return type}] \leavevmode
function

\end{description}\end{quote}

\end{fulllineitems}

\index{logout\_required() (in module config.decorators)@\spxentry{logout\_required()}\spxextra{in module config.decorators}}

\begin{fulllineitems}
\phantomsection\label{\detokenize{index:config.decorators.logout_required}}\pysiglinewithargsret{\sphinxcode{\sphinxupquote{config.decorators.}}\sphinxbfcode{\sphinxupquote{logout\_required}}}{\emph{\DUrole{n}{f}}}{}
logout\_required Ensures that a user is logged out before granting access to non user restricted pages

Will check if a user is logged out and only allow a redirect to the non user restricted page if there is
no user logged in. If a user is logged in it will simply log them out automatically before redirecting to
the requested endpoint. No logout action on the part of the user is necessary.
NOTE: This function should never be called as a function, only used as a decorator above functions
who require its functionality to be implemented on entry. Always use a decorator (i.e. @logout\_required)
to invoke this function.
\begin{quote}\begin{description}
\item[{Parameters}] \leavevmode
\sphinxstyleliteralstrong{\sphinxupquote{f}} (\sphinxstyleliteralemphasis{\sphinxupquote{function}}) \textendash{} The function being decorated. This will be automatically filled when using the @ tag

\item[{Returns}] \leavevmode
The wrap function, whose contents are the input function, modified to add the functionality of this decorator

\item[{Return type}] \leavevmode
function

\end{description}\end{quote}

\end{fulllineitems}



\chapter{Data Database}
\label{\detokenize{index:module-data.database}}\label{\detokenize{index:data-database}}\index{module@\spxentry{module}!data.database@\spxentry{data.database}}\index{data.database@\spxentry{data.database}!module@\spxentry{module}}

\section{Database Library}
\label{\detokenize{index:database-library}}
A collection of functions capable of interacting with
a sqlite3 single file databse.
NOTE: Users are returned as lists, whose entries are
in the following order
\begin{quote}

{[} \_userid,
username,
password,
LowerRateLimit,
UpperRateLimit,
AtrialAmplitude,
AtrialPulseWidth,
AtrialRefractoryPeriod,
VentricularAmplitude,
VentricularPulseWidth,
VentricularRefractoryPeriod {]}
\end{quote}
\index{find\_user() (in module data.database)@\spxentry{find\_user()}\spxextra{in module data.database}}

\begin{fulllineitems}
\phantomsection\label{\detokenize{index:data.database.find_user}}\pysiglinewithargsret{\sphinxcode{\sphinxupquote{data.database.}}\sphinxbfcode{\sphinxupquote{find\_user}}}{\emph{\DUrole{n}{cursor}}, \emph{\DUrole{n}{username}\DUrole{o}{=}\DUrole{default_value}{None}}, \emph{\DUrole{n}{password}\DUrole{o}{=}\DUrole{default_value}{None}}}{}
find\_user Given search parameters will find all matching users in the database

Given one or more of the optional search parameters will return a list of all users
matching that search criteria. Accepted search parameters are username and password.
If neither optional parameters are given, the function will return None
\begin{quote}\begin{description}
\item[{Parameters}] \leavevmode\begin{itemize}
\item {} 
\sphinxstyleliteralstrong{\sphinxupquote{cursor}} (\sphinxcode{\sphinxupquote{sqlite3.Cursor}}) \textendash{} The cursor handler for the database to search for the user in

\item {} 
\sphinxstyleliteralstrong{\sphinxupquote{username}} (\sphinxstyleliteralemphasis{\sphinxupquote{str}}\sphinxstyleliteralemphasis{\sphinxupquote{, }}\sphinxstyleliteralemphasis{\sphinxupquote{optional}}) \textendash{} The username of the user to search for, defaults to None

\item {} 
\sphinxstyleliteralstrong{\sphinxupquote{password}} (\sphinxstyleliteralemphasis{\sphinxupquote{str}}\sphinxstyleliteralemphasis{\sphinxupquote{, }}\sphinxstyleliteralemphasis{\sphinxupquote{optional}}) \textendash{} The password of the user to search for, defaults to None

\end{itemize}

\item[{Returns}] \leavevmode
A list of tuples containing all users matching the search query

\item[{Return type}] \leavevmode
list

\end{description}\end{quote}

\end{fulllineitems}

\index{get\_rows() (in module data.database)@\spxentry{get\_rows()}\spxextra{in module data.database}}

\begin{fulllineitems}
\phantomsection\label{\detokenize{index:data.database.get_rows}}\pysiglinewithargsret{\sphinxcode{\sphinxupquote{data.database.}}\sphinxbfcode{\sphinxupquote{get\_rows}}}{\emph{\DUrole{n}{cursor}}}{}
get\_rows Returns the number of rows (.i.e users) in the database
\begin{quote}\begin{description}
\item[{Parameters}] \leavevmode
\sphinxstyleliteralstrong{\sphinxupquote{cursor}} (\sphinxcode{\sphinxupquote{sqlite3.Cursor}}) \textendash{} The cursor handler for the database

\item[{Returns}] \leavevmode
the number of rows contained in the user table of the database

\item[{Return type}] \leavevmode
int

\end{description}\end{quote}

\end{fulllineitems}

\index{get\_user() (in module data.database)@\spxentry{get\_user()}\spxextra{in module data.database}}

\begin{fulllineitems}
\phantomsection\label{\detokenize{index:data.database.get_user}}\pysiglinewithargsret{\sphinxcode{\sphinxupquote{data.database.}}\sphinxbfcode{\sphinxupquote{get\_user}}}{\emph{\DUrole{n}{cursor}}, \emph{\DUrole{n}{id}}}{}
get\_user Returns a complete users information given their unique ID

Return type is a list of users. Since the search is done by unique ID,
this list is garunteed to be either of length one, if a user with matching
unique ID is found, or zero, if no user with matching unique ID is found.
\begin{quote}\begin{description}
\item[{Parameters}] \leavevmode\begin{itemize}
\item {} 
\sphinxstyleliteralstrong{\sphinxupquote{cursor}} (\sphinxcode{\sphinxupquote{sqlite3.Cursor}}) \textendash{} The cursor handler for the database the user can be found in

\item {} 
\sphinxstyleliteralstrong{\sphinxupquote{id}} (\sphinxstyleliteralemphasis{\sphinxupquote{int}}) \textendash{} The unique ID of the user to search for

\end{itemize}

\item[{Returns}] \leavevmode
A list of tuples containing the contents of the first item matching the search query

\item[{Return type}] \leavevmode
list

\end{description}\end{quote}

\end{fulllineitems}

\index{get\_user\_history() (in module data.database)@\spxentry{get\_user\_history()}\spxextra{in module data.database}}

\begin{fulllineitems}
\phantomsection\label{\detokenize{index:data.database.get_user_history}}\pysiglinewithargsret{\sphinxcode{\sphinxupquote{data.database.}}\sphinxbfcode{\sphinxupquote{get\_user\_history}}}{\emph{\DUrole{n}{cursor}}, \emph{\DUrole{n}{id}}}{}
get\_user\_history Returns a complete list of all the users past parameters

Return type is a list of all past parameters, including the current ones.
This function has no garunteed list size, as it depends on how many history
entries the user has made.
\begin{quote}\begin{description}
\item[{Parameters}] \leavevmode\begin{itemize}
\item {} 
\sphinxstyleliteralstrong{\sphinxupquote{cursor}} (\sphinxcode{\sphinxupquote{sqlite3.Cursor}}) \textendash{} The cursor handler for the database the user can be found in

\item {} 
\sphinxstyleliteralstrong{\sphinxupquote{id}} (\sphinxstyleliteralemphasis{\sphinxupquote{int}}) \textendash{} The unique ID of the user to search for

\end{itemize}

\item[{Returns}] \leavevmode
A list of all the users past pacemaker parameters

\item[{Return type}] \leavevmode
list

\end{description}\end{quote}

\end{fulllineitems}

\index{get\_user\_parameters() (in module data.database)@\spxentry{get\_user\_parameters()}\spxextra{in module data.database}}

\begin{fulllineitems}
\phantomsection\label{\detokenize{index:data.database.get_user_parameters}}\pysiglinewithargsret{\sphinxcode{\sphinxupquote{data.database.}}\sphinxbfcode{\sphinxupquote{get\_user\_parameters}}}{\emph{\DUrole{n}{cursor}}, \emph{\DUrole{n}{id}}}{}
get\_user\_parameters Returns a complete list of the users most recent parameters

Return type is a list of parameters. Since the search is done by unique ID,
this list is garunteed to be of constant length, defined by the parameter list
in the application configuration.
\begin{quote}\begin{description}
\item[{Parameters}] \leavevmode\begin{itemize}
\item {} 
\sphinxstyleliteralstrong{\sphinxupquote{cursor}} (\sphinxcode{\sphinxupquote{sqlite3.Cursor}}) \textendash{} The cursor handler for the database the user can be found in

\item {} 
\sphinxstyleliteralstrong{\sphinxupquote{id}} (\sphinxstyleliteralemphasis{\sphinxupquote{int}}) \textendash{} The unique ID of the user to search for

\end{itemize}

\item[{Returns}] \leavevmode
A list of the users current pacemaker parameters

\item[{Return type}] \leavevmode
list

\end{description}\end{quote}

\end{fulllineitems}

\index{init\_db() (in module data.database)@\spxentry{init\_db()}\spxextra{in module data.database}}

\begin{fulllineitems}
\phantomsection\label{\detokenize{index:data.database.init_db}}\pysiglinewithargsret{\sphinxcode{\sphinxupquote{data.database.}}\sphinxbfcode{\sphinxupquote{init\_db}}}{\emph{\DUrole{n}{file}}}{}
init\_db Initializes a database located at a given file location

The file location should be specified relative to the \textasciitilde{}/3K04\sphinxhyphen{}Pacemaker/DCM/flaskapp/data
directory. The file should also have a supported sqlite3 extension (.db .db3 .sdb .s3db
.sqlite .sqlite3) and if the file does not already exist it will be created and
populated with a new databases.
\begin{quote}\begin{description}
\item[{Parameters}] \leavevmode
\sphinxstyleliteralstrong{\sphinxupquote{file}} (\sphinxstyleliteralemphasis{\sphinxupquote{str}}) \textendash{} The relative file location of the single file sqlite3 database

\item[{Returns}] \leavevmode
A tuple containing the databases connection handler and cursor (sqlite3.Connection, sqlite3.Cursor)

\item[{Return type}] \leavevmode
tuple

\end{description}\end{quote}

\end{fulllineitems}

\index{insert\_user() (in module data.database)@\spxentry{insert\_user()}\spxextra{in module data.database}}

\begin{fulllineitems}
\phantomsection\label{\detokenize{index:data.database.insert_user}}\pysiglinewithargsret{\sphinxcode{\sphinxupquote{data.database.}}\sphinxbfcode{\sphinxupquote{insert\_user}}}{\emph{\DUrole{n}{conn}}, \emph{\DUrole{n}{cursor}}, \emph{\DUrole{n}{username}}, \emph{\DUrole{n}{password}}}{}
insert\_user Given a username and password of a new user, will insert the user into the database

This function will create a new entry in the database of a user with the given username and password.
Only the users username and password are initialized upon user creation, the pacemaker parameters will
default to None, forcing the user to manually enter their parameters.
NOTE: This function does no check for conflicting users in the database before inserting a new user.
It is up to the user of this function to check for conflicts (if they wish to do so) before calling
this function.
\begin{quote}\begin{description}
\item[{Parameters}] \leavevmode\begin{itemize}
\item {} 
\sphinxstyleliteralstrong{\sphinxupquote{conn}} (\sphinxcode{\sphinxupquote{sqlite3.Connection}}) \textendash{} The connection handler for the database to insert a new user into

\item {} 
\sphinxstyleliteralstrong{\sphinxupquote{cursor}} (\sphinxcode{\sphinxupquote{sqlite3.Cursor}}) \textendash{} The cursor handler for the database to insert a new user into

\item {} 
\sphinxstyleliteralstrong{\sphinxupquote{username}} (\sphinxstyleliteralemphasis{\sphinxupquote{str}}) \textendash{} The username for the new user

\item {} 
\sphinxstyleliteralstrong{\sphinxupquote{password}} (\sphinxstyleliteralemphasis{\sphinxupquote{str}}) \textendash{} The password for the new user

\end{itemize}

\end{description}\end{quote}

\end{fulllineitems}

\index{update\_pacemaker\_parameters() (in module data.database)@\spxentry{update\_pacemaker\_parameters()}\spxextra{in module data.database}}

\begin{fulllineitems}
\phantomsection\label{\detokenize{index:data.database.update_pacemaker_parameters}}\pysiglinewithargsret{\sphinxcode{\sphinxupquote{data.database.}}\sphinxbfcode{\sphinxupquote{update\_pacemaker\_parameters}}}{\emph{\DUrole{n}{conn}}, \emph{\DUrole{n}{cursor}}, \emph{\DUrole{n}{id}}, \emph{\DUrole{n}{values}}}{}
update\_pacemaker\_parameters Given a list of pacemaker parameters, updates the database values

When given handler to the database and the unique ID of the user being affected, will update the
users pacemaker parameters to match the input list.
NOTE: No complete check is done to ensure the validity of the input, it is up
to the method user to ensure the lists correctness.
\begin{quote}\begin{description}
\item[{Parameters}] \leavevmode\begin{itemize}
\item {} 
\sphinxstyleliteralstrong{\sphinxupquote{conn}} (\sphinxcode{\sphinxupquote{sqlite3.Connection}}) \textendash{} The connection handler for the database whos contents to change

\item {} 
\sphinxstyleliteralstrong{\sphinxupquote{cursor}} (\sphinxcode{\sphinxupquote{sqlite3.Cursor}}) \textendash{} The cursor handler for the database whos contents to change

\item {} 
\sphinxstyleliteralstrong{\sphinxupquote{id}} (\sphinxstyleliteralemphasis{\sphinxupquote{int}}) \textendash{} The unique ID of the user whos parameters should be changed

\item {} 
\sphinxstyleliteralstrong{\sphinxupquote{values}} (\sphinxstyleliteralemphasis{\sphinxupquote{list}}) \textendash{} A list of pacemaker parameters, whos order matches the databases contents

\end{itemize}

\end{description}\end{quote}

\end{fulllineitems}



\chapter{Data User}
\label{\detokenize{index:module-data.user}}\label{\detokenize{index:data-user}}\index{module@\spxentry{module}!data.user@\spxentry{data.user}}\index{data.user@\spxentry{data.user}!module@\spxentry{module}}

\section{User Class}
\label{\detokenize{index:user-class}}
The class to represent a user of this application.
Capable of initializing and accessing the database
specified in the application configuration, as well
as loggin in and out, creating an account, and
modifying the pacemaker parameters of the currently
logged in user.
\index{User (class in data.user)@\spxentry{User}\spxextra{class in data.user}}

\begin{fulllineitems}
\phantomsection\label{\detokenize{index:data.user.User}}\pysigline{\sphinxbfcode{\sphinxupquote{class }}\sphinxcode{\sphinxupquote{data.user.}}\sphinxbfcode{\sphinxupquote{User}}}
This is a class representation of a simple flask app user

The class to represent a user of this application.
Capable of initializing and accessing the database
specified in the application configuration, as well
as loggin in and out, creating an account, and
modifying the pacemaker parameters of the currently
logged in user.
\begin{quote}\begin{description}
\item[{Parameters}] \leavevmode
\sphinxstyleliteralstrong{\sphinxupquote{config}} (class:\sphinxtitleref{configparser.ConfigParser}) \textendash{} A handle to the \sphinxcode{\sphinxupquote{configparser.ConfigParser}} config
object initialized by the main application on startup

\end{description}\end{quote}
\index{create\_account() (data.user.User method)@\spxentry{create\_account()}\spxextra{data.user.User method}}

\begin{fulllineitems}
\phantomsection\label{\detokenize{index:data.user.User.create_account}}\pysiglinewithargsret{\sphinxbfcode{\sphinxupquote{create\_account}}}{\emph{\DUrole{n}{username}}, \emph{\DUrole{n}{password}}}{}
create\_account Creates a new user account

Checks the database to ensure no user with the same username exists,
and that the maximum allowable local\sphinxhyphen{}agents has not been exceeded
(defined in the application.ini). If no conflicts exist, a new user
is created then both inserted into the database and logged in.
\begin{quote}\begin{description}
\item[{Parameters}] \leavevmode\begin{itemize}
\item {} 
\sphinxstyleliteralstrong{\sphinxupquote{username}} (\sphinxstyleliteralemphasis{\sphinxupquote{str}}) \textendash{} The username of the new user being created

\item {} 
\sphinxstyleliteralstrong{\sphinxupquote{password}} (\sphinxstyleliteralemphasis{\sphinxupquote{str}}) \textendash{} The password of the new user being created

\end{itemize}

\item[{Returns}] \leavevmode
True if the account creation was successful, False otherwise

\item[{Return type}] \leavevmode
bool

\end{description}\end{quote}

\end{fulllineitems}

\index{create\_history\_file() (data.user.User method)@\spxentry{create\_history\_file()}\spxextra{data.user.User method}}

\begin{fulllineitems}
\phantomsection\label{\detokenize{index:data.user.User.create_history_file}}\pysiglinewithargsret{\sphinxbfcode{\sphinxupquote{create\_history\_file}}}{}{}
create\_history\_file Creates a csv file containing the users parameter history

\end{fulllineitems}

\index{get\_history() (data.user.User method)@\spxentry{get\_history()}\spxextra{data.user.User method}}

\begin{fulllineitems}
\phantomsection\label{\detokenize{index:data.user.User.get_history}}\pysiglinewithargsret{\sphinxbfcode{\sphinxupquote{get\_history}}}{}{}
get\_history Get the users history

Gets the users curretn history by calling the database
helper function
\begin{quote}\begin{description}
\item[{Returns}] \leavevmode
A list fo the users history, including the current pacemaker values

\item[{Return type}] \leavevmode
list

\end{description}\end{quote}

\end{fulllineitems}

\index{get\_limits() (data.user.User method)@\spxentry{get\_limits()}\spxextra{data.user.User method}}

\begin{fulllineitems}
\phantomsection\label{\detokenize{index:data.user.User.get_limits}}\pysiglinewithargsret{\sphinxbfcode{\sphinxupquote{get\_limits}}}{}{}
get\_limits Get the users parameter limits

Returns a dictionary of all the users parameter limits
as they were defined in the application configuration
\begin{quote}\begin{description}
\item[{Returns}] \leavevmode
A dictionary of the users parameter limits

\item[{Return type}] \leavevmode
dict

\end{description}\end{quote}

\end{fulllineitems}

\index{get\_pacemaker\_mode() (data.user.User method)@\spxentry{get\_pacemaker\_mode()}\spxextra{data.user.User method}}

\begin{fulllineitems}
\phantomsection\label{\detokenize{index:data.user.User.get_pacemaker_mode}}\pysiglinewithargsret{\sphinxbfcode{\sphinxupquote{get\_pacemaker\_mode}}}{}{}
get\_pacemaker\_mode Returns the current pacemaker mode
\begin{quote}\begin{description}
\item[{Returns}] \leavevmode
The current pacemaker mode, can be either a String or None

\item[{Return type}] \leavevmode
str

\end{description}\end{quote}

\end{fulllineitems}

\index{get\_pacemaker\_parameters() (data.user.User method)@\spxentry{get\_pacemaker\_parameters()}\spxextra{data.user.User method}}

\begin{fulllineitems}
\phantomsection\label{\detokenize{index:data.user.User.get_pacemaker_parameters}}\pysiglinewithargsret{\sphinxbfcode{\sphinxupquote{get\_pacemaker\_parameters}}}{}{}
get\_pacemaker\_parameters Returns a dictionary of this users pacemaker parameters
\begin{quote}\begin{description}
\item[{Returns}] \leavevmode
A dictionary of this users pacemaker parameters

\item[{Return type}] \leavevmode
dict

\end{description}\end{quote}

\end{fulllineitems}

\index{get\_username() (data.user.User method)@\spxentry{get\_username()}\spxextra{data.user.User method}}

\begin{fulllineitems}
\phantomsection\label{\detokenize{index:data.user.User.get_username}}\pysiglinewithargsret{\sphinxbfcode{\sphinxupquote{get\_username}}}{}{}
get\_username Returns this users username
\begin{quote}\begin{description}
\item[{Returns}] \leavevmode
This users username

\item[{Return type}] \leavevmode
str

\end{description}\end{quote}

\end{fulllineitems}

\index{is\_loggedin() (data.user.User method)@\spxentry{is\_loggedin()}\spxextra{data.user.User method}}

\begin{fulllineitems}
\phantomsection\label{\detokenize{index:data.user.User.is_loggedin}}\pysiglinewithargsret{\sphinxbfcode{\sphinxupquote{is\_loggedin}}}{}{}
is\_loggedin Checks if this user is logged in
\begin{quote}\begin{description}
\item[{Returns}] \leavevmode
True if the user is logged in, False otherwise

\item[{Return type}] \leavevmode
bool

\end{description}\end{quote}

\end{fulllineitems}

\index{login() (data.user.User method)@\spxentry{login()}\spxextra{data.user.User method}}

\begin{fulllineitems}
\phantomsection\label{\detokenize{index:data.user.User.login}}\pysiglinewithargsret{\sphinxbfcode{\sphinxupquote{login}}}{\emph{\DUrole{n}{username}}, \emph{\DUrole{n}{password}}}{}
login Attempts to log a user in, given their username and password

Searches the database to check if the user with matching username and
password exists. No conflict management (i.e. ensuring only one user
matches that username) is necessary since it is handled on account
creation. If a matching user is found the {\hyperref[\detokenize{index:module-data.user}]{\sphinxcrossref{\sphinxcode{\sphinxupquote{data.user}}}}} is updated
with that users information and the user is logged in.
\begin{quote}\begin{description}
\item[{Parameters}] \leavevmode\begin{itemize}
\item {} 
\sphinxstyleliteralstrong{\sphinxupquote{username}} (\sphinxstyleliteralemphasis{\sphinxupquote{str}}) \textendash{} The username of the user trying to login

\item {} 
\sphinxstyleliteralstrong{\sphinxupquote{password}} (\sphinxstyleliteralemphasis{\sphinxupquote{str}}) \textendash{} The password of the user trying to login

\end{itemize}

\end{description}\end{quote}

\end{fulllineitems}

\index{logout() (data.user.User method)@\spxentry{logout()}\spxextra{data.user.User method}}

\begin{fulllineitems}
\phantomsection\label{\detokenize{index:data.user.User.logout}}\pysiglinewithargsret{\sphinxbfcode{\sphinxupquote{logout}}}{}{}
logout Logs out the currently logged in user

\end{fulllineitems}

\index{update\_all\_pacemaker\_parameters() (data.user.User method)@\spxentry{update\_all\_pacemaker\_parameters()}\spxextra{data.user.User method}}

\begin{fulllineitems}
\phantomsection\label{\detokenize{index:data.user.User.update_all_pacemaker_parameters}}\pysiglinewithargsret{\sphinxbfcode{\sphinxupquote{update\_all\_pacemaker\_parameters}}}{\emph{\DUrole{n}{values}}}{}
update\_all\_pacemaker\_parameters Updates all pacemaker parameters

Given a list of pacemaker parameters, in the same order as the values of
the pacemaker parameters dictionary, will update every value of the dictionary.
NOTE: No complete check is done to ensure the validity of the input, it is up
to the method user to ensure the lists correctness.
\begin{quote}\begin{description}
\item[{Parameters}] \leavevmode
\sphinxstyleliteralstrong{\sphinxupquote{values}} (\sphinxstyleliteralemphasis{\sphinxupquote{list}}) \textendash{} An ordered list of the updated pacemaker parameters

\item[{Returns}] \leavevmode
True if the pacemaker parameters were updated sucessfully, False otherwise

\item[{Return type}] \leavevmode
bool

\end{description}\end{quote}

\end{fulllineitems}

\index{update\_pacemaker\_mode() (data.user.User method)@\spxentry{update\_pacemaker\_mode()}\spxextra{data.user.User method}}

\begin{fulllineitems}
\phantomsection\label{\detokenize{index:data.user.User.update_pacemaker_mode}}\pysiglinewithargsret{\sphinxbfcode{\sphinxupquote{update\_pacemaker\_mode}}}{\emph{\DUrole{n}{mode}}}{}
update\_pacemaker\_mode Changes the pacemaker mode

Given a mode that is contained in the application
configurations list of allowed modes, will change
the current pacemaker mode to the new one. If no
valid mode is given will do nothing and return
False.
\begin{quote}\begin{description}
\item[{Parameters}] \leavevmode
\sphinxstyleliteralstrong{\sphinxupquote{mode}} (\sphinxstyleliteralemphasis{\sphinxupquote{str}}) \textendash{} The pacemaker mode to change to

\item[{Returns}] \leavevmode
True if the pacemaker mode was successfully changed, Fale otherwise

\item[{Return type}] \leavevmode
bool

\end{description}\end{quote}

\end{fulllineitems}

\index{update\_pacemaker\_parameter() (data.user.User method)@\spxentry{update\_pacemaker\_parameter()}\spxextra{data.user.User method}}

\begin{fulllineitems}
\phantomsection\label{\detokenize{index:data.user.User.update_pacemaker_parameter}}\pysiglinewithargsret{\sphinxbfcode{\sphinxupquote{update\_pacemaker\_parameter}}}{\emph{\DUrole{n}{key}}, \emph{\DUrole{n}{value}}}{}
update\_pacemaker\_parameter Updates a single pacemaker parameter

Given a valid key (one already contained in the pacemaker parameters
dictionary), will update the value of that key with the passes in
value.
\begin{quote}\begin{description}
\item[{Parameters}] \leavevmode\begin{itemize}
\item {} 
\sphinxstyleliteralstrong{\sphinxupquote{key}} (\sphinxstyleliteralemphasis{\sphinxupquote{str}}) \textendash{} A key already contained in the pacemaker parameters dictionary

\item {} 
\sphinxstyleliteralstrong{\sphinxupquote{value}} (\sphinxstyleliteralemphasis{\sphinxupquote{int}}) \textendash{} An updated value for the associated key

\end{itemize}

\item[{Returns}] \leavevmode
True if the parameter was sucessfully updated, False otherwise

\item[{Return type}] \leavevmode
bool

\end{description}\end{quote}

\end{fulllineitems}


\end{fulllineitems}



\chapter{Graphs Graphing}
\label{\detokenize{index:module-graphs.graphing}}\label{\detokenize{index:graphs-graphing}}\index{module@\spxentry{module}!graphs.graphing@\spxentry{graphs.graphing}}\index{graphs.graphing@\spxentry{graphs.graphing}!module@\spxentry{module}}

\section{Graphing Library}
\label{\detokenize{index:graphing-library}}
A collection of graphiung functions
used by the app to generate both data
points to be rendered live, as well as
publish a graphs history to a csv file.
\index{publish\_data() (in module graphs.graphing)@\spxentry{publish\_data()}\spxextra{in module graphs.graphing}}

\begin{fulllineitems}
\phantomsection\label{\detokenize{index:graphs.graphing.publish_data}}\pysiglinewithargsret{\sphinxcode{\sphinxupquote{graphs.graphing.}}\sphinxbfcode{\sphinxupquote{publish\_data}}}{\emph{\DUrole{n}{username}}}{}
publish\_data Publishes the current (or most recent) live graphs data to a csv file

Will generate a csv file containing all the data from when the current (or most recent)
graph was started.

\end{fulllineitems}

\index{random() (in module graphs.graphing)@\spxentry{random()}\spxextra{in module graphs.graphing}}

\begin{fulllineitems}
\phantomsection\label{\detokenize{index:graphs.graphing.random}}\pysiglinewithargsret{\sphinxcode{\sphinxupquote{graphs.graphing.}}\sphinxbfcode{\sphinxupquote{random}}}{}{{ $\rightarrow$ x in the interval {[}0, 1).}}
\end{fulllineitems}

\index{set\_start\_time() (in module graphs.graphing)@\spxentry{set\_start\_time()}\spxextra{in module graphs.graphing}}

\begin{fulllineitems}
\phantomsection\label{\detokenize{index:graphs.graphing.set_start_time}}\pysiglinewithargsret{\sphinxcode{\sphinxupquote{graphs.graphing.}}\sphinxbfcode{\sphinxupquote{set\_start\_time}}}{}{}
set\_start\_time Sets the start time for the graph

Called when starting a new graph, will reset the running
timer allowing the new graph to start at the zero value.

\end{fulllineitems}

\index{temp\_serial\_placeholder() (in module graphs.graphing)@\spxentry{temp\_serial\_placeholder()}\spxextra{in module graphs.graphing}}

\begin{fulllineitems}
\phantomsection\label{\detokenize{index:graphs.graphing.temp_serial_placeholder}}\pysiglinewithargsret{\sphinxcode{\sphinxupquote{graphs.graphing.}}\sphinxbfcode{\sphinxupquote{temp\_serial\_placeholder}}}{}{}
temp\_serial\_placeholder Placeholder for serial functionality

Generates a sine and cosine point output to be rendered to the
live graphs to test their functionality without relying on the
serial communications.
\begin{quote}\begin{description}
\item[{Returns}] \leavevmode
A list containing a sine and cosine point in this format {[} sine(current\_time), cos(current\_time){]}

\item[{Return type}] \leavevmode
list

\end{description}\end{quote}

\end{fulllineitems}

\index{update\_data() (in module graphs.graphing)@\spxentry{update\_data()}\spxextra{in module graphs.graphing}}

\begin{fulllineitems}
\phantomsection\label{\detokenize{index:graphs.graphing.update_data}}\pysiglinewithargsret{\sphinxcode{\sphinxupquote{graphs.graphing.}}\sphinxbfcode{\sphinxupquote{update\_data}}}{}{}
update\_data Returns a new data point for the graph

Called to request a new data point for the live graph.
Will return a list of lists, each internal list containing
a point with a timestamp (x\sphinxhyphen{}value) and parameter value (y\sphinxhyphen{}value).
The timestamp is determined by the time difference between ‘now’ and
when the set\_start\_time() function was last called.
\begin{quote}\begin{description}
\item[{Returns}] \leavevmode
A list of lists containing a new point to be added to each line being rendered by the live graph

\item[{Return type}] \leavevmode
list

\end{description}\end{quote}

\end{fulllineitems}



\chapter{Tests Tests}
\label{\detokenize{index:module-tests.tests}}\label{\detokenize{index:tests-tests}}\index{module@\spxentry{module}!tests.tests@\spxentry{tests.tests}}\index{tests.tests@\spxentry{tests.tests}!module@\spxentry{module}}

\chapter{Indices and tables}
\label{\detokenize{index:indices-and-tables}}\begin{itemize}
\item {} 
\DUrole{xref,std,std-ref}{genindex}

\item {} 
\DUrole{xref,std,std-ref}{modindex}

\item {} 
\DUrole{xref,std,std-ref}{search}

\end{itemize}


\renewcommand{\indexname}{Python Module Index}
\begin{sphinxtheindex}
\let\bigletter\sphinxstyleindexlettergroup
\bigletter{a}
\item\relax\sphinxstyleindexentry{app}\sphinxstyleindexpageref{index:\detokenize{module-app}}
\indexspace
\bigletter{c}
\item\relax\sphinxstyleindexentry{config.config\_manager}\sphinxstyleindexpageref{index:\detokenize{module-config.config_manager}}
\item\relax\sphinxstyleindexentry{config.decorators}\sphinxstyleindexpageref{index:\detokenize{module-config.decorators}}
\indexspace
\bigletter{d}
\item\relax\sphinxstyleindexentry{data.database}\sphinxstyleindexpageref{index:\detokenize{module-data.database}}
\item\relax\sphinxstyleindexentry{data.user}\sphinxstyleindexpageref{index:\detokenize{module-data.user}}
\indexspace
\bigletter{g}
\item\relax\sphinxstyleindexentry{graphs.graphing}\sphinxstyleindexpageref{index:\detokenize{module-graphs.graphing}}
\indexspace
\bigletter{t}
\item\relax\sphinxstyleindexentry{tests.tests}\sphinxstyleindexpageref{index:\detokenize{module-tests.tests}}
\end{sphinxtheindex}

\renewcommand{\indexname}{Index}
\printindex
\end{document}